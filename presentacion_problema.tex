\chapter{Presentación del problema}

Para cumplir con el objetivo propuesto, se dividirá el desarrollo del sistema de generación de texto en dos problemas. El primero consiste en crear un módulo que preprocese la entrada para organizar y estructurar la información de forma más significativa. Reiter E. ha propuesto esta etapa previa de preprocesamiento~\cite{reiter2007architecture} para el análisis y la interpretación de datos, aunque fue aplicada en una entrada de datos en bruto (generada directamente de sensores) en lugar de una base de conocimiento.

El segundo problema consiste en diseñar e implementar el módulo que toma como entrada la información organizada y la convierte en un documento de texto. Para esto usaremos como referencia las tareas descritas en la Sección~\ref{sec:tareas_gnl}.

Se puede apreciar la composición de nuestro sistema de generación de lenguaje natural en la Figura~\ref{fig:modulos_sgln}.

\begin{figure}
    \centering
    \includegraphics[width=12cm, height=4cm]{img/presentacion_problema/modulos_sgln.pdf}
    \caption{Módulos que componen el sistema de generación de lenguaje natural}
    \label{fig:modulos_sgln}
\end{figure}

\section{Presentación del problema de la organización de la información para la coherencia del texto}
\label{sec:problema_coherencia-texto}

Basándonos en la Teoría de Centrado, tenemos una base para elegir cómo presentar en lenguaje natural y de manera coherente la información de cada entidad contenida en la ontología. 

Dado una entidad, se puede recorrer la estructura de grafo asociada a la ontología para recuperar la información necesaria para describir esa entidad. Sin embargo, aún queda decidir cómo recorrer el grafo, ya que cada recorrido genera un resultado diferente. 

Como hipótesis, consideramos que para mantener una coherencia local, es necesario recorrer el grafo en anchura, obteniendo la máxima cantidad de información posible acerca de una única entidad \emph{E}, de manera que el foco de atención solo salte hacia otras entidades cuando no haya nada más para decir acerca de \emph{E}.

En cuanto a la coherencia global, podemos valernos de la estructura del grafo, para medir el grado de relación entre las entidades, aún antes del proceso de selección de información para describir las entidades. Se pueden utilizar diferentes criterios para medir el grado de relación. A continuación veremos un ejemplo a partir de la Figura~\ref{fig:pizza.owl}, la cual contiene representada gráficamente un subconjunto de la ontología pizza.owl\footnote{https://protege.stanford.edu/ontologies/pizza/pizza.owl}.

\begin{figure}
\centering
\subfigure[Jerarquía de clases de la ontología pizza]{\includegraphics[width=70mm]{img/presentacion_problema/onto_pizza.pdf}}
\subfigure[Jerarquía de propiedades de la ontología pizza]{\includegraphics[width=3cm]{img/presentacion_problema/onto_pizza_properties.pdf}}
\caption{Subconjunto de la ontología de pizzas.} \label{fig:pizza.owl}
\end{figure}

Aún antes de generar las proposiciones que reflejan la información del grafo, se puede ver que hay formas más adecuadas que otras para comenzar a transmitirle a un receptor toda la información del grafo. Algunos de los posibles tópicos que engloban o relacionan a la mayoría de los datos expuestos son ``Tipo de comidas'', o ``Ingredientes de la pizza'', siendo que, la mayor cantidad de información está relacionada a la pizza. 

Para identificar los nodos más relevantes, pueden usarse la disposición jerárquica del grafo o las medidas de centralidad de los nodos.

También se puede hacer uso de la jerarquía de las propiedades, para reconocer su tipo más general a través de la relación de generalización, como en el caso de las relaciones \emph{tiene cobertura} y \emph{tiene base}, con el fin de referirse a ellas como ``Ingredientes''.

En estos ejemplos hemos presentado dos problemáticas a tratar: la obtención de los tópicos más relevantes para tratar en el texto (para abordar la coherencia global), que en nuestro caso son unidades de información representadas por \emph{owl:Class} y \emph{owl:NamedIndividual}; y el recorrido del grafo para obtener las unidades de información adecuada que maximicen la relación semántica respecto a los tópicos del texto (para abordar la cohesión local). En este caso, las unidades de información que se tienen en cuenta para el recorrido del grafo son \emph{owl:Class}, \emph{owl:ObjectProperty} y \emph{owl:DataProperty}.
 
\section{Presentación del problema de verbalización}
%Como presentamos en \ref{sec:tareas_gnl}, existen ciertas tareas a desarrollar en un sistema de generación de texto.
Como presentamos en el Objetivo, se esperan alcanzar dos cualidades principales en el texto de salida:
\begin{enumerate}
    \item A nivel macro, el texto debe tener una estructura que permita visualizar los principales temas, presentándolos de manera aislada (secciones, párrafos), pero que en conjunto se complementen para que el lector pueda establecer las relaciones adecuadas entre ellos. Como objetivo principal se espera alcanzar una estructura que presente la información más importante y necesaria cuanto antes, preparando al lector para que pueda comprender información nueva.
    \item A nivel micro, el texto debe cumplir con cierta fluidez, utilizando frases que tengan una gramática aceptable, tratando de maximizar la cohesión.
\end{enumerate}{}