\chapter{Antecedentes conceptuales}

\section{Trabajos relacionados}
%%copiado y pegado de la propuesta

El uso de ontologías para la descripción de dominios se ha ido intensificando en los últimos años. La utilidad de las ontologías radica en la capacidad de compartir e integrar datos, y del potencial de inferir conocimiento a través de agentes informáticos. Sin embargo, el desarrollo y análisis de ontologías resulta muy difícil para los seres humanos, tanto para usuarios expertos como para los que no son expertos en ontologías, por lo que es imprescindible el uso de herramientas de soporte para facilitar el trabajo con las mismas. Actualmente existen algunas aplicaciones para desarrollar y explorar ontologías, como Protégé~\cite{protege}, Mindswap~\cite{golbeck2002new}, OntoEdit~\cite{sure2002ontoedit}, OntoLingua \cite{farquhar1997ontolingua}, WebOnto~\cite{domingue1998tadzebao}, entre otras, pero no resultan suficientemente satisfactorias para usuarios no expertos, o usuarios finales. 

Varios trabajos involucran el procesamiento de lenguaje natural en el ámbito de las ontologías. Por un lado, existen herramientas que se enfocan en la verbalización de un conjunto de axiomas pertenecientes a una ontología, con el fin de generar un conjunto de sentencias que puedan ser utilizadas para reconocer la información perteneciente al dominio modelado, o documentar la ontología, utilizando lenguaje natural. En este tipo de trabajos se encuentran algunas herramientas como OntoVerbal~\cite{liang2013ontoverbal}, que utiliza la frecuencia con la que se utilizan conjuntos de axiomas para la descripción de entidades, y describe patrones lingüísticos para verbalizar los conjuntos de mayor frecuencia. Además, agrupa los axiomas alrededor de un tópico para crear párrafos. Otra herramienta es NaturalOWL~\cite{galanis2007generating}, que se enfoca en la flexibilidad y fluidez del texto, agregando información lingüística a la ontología para verbalizar los axiomas.

Por otro lado, se han desarrollado herramientas para asistir durante el desarrollo de la ontología, permitiendo acceder a la información modelada en lógicas descriptivas en un formato de texto, y poder editarlo en lenguaje natural. Entre las herramientas desarrolladas se encuentran CLOnE~\cite{power2010complexity} y Aqualog~\cite{lopez2005aqualog},completamente orientadas a la comprensión de texto y no a la generación; y Attempto Tools~\cite{attempto}, que utiliza Attempto Controlled English  (ACE) que es un lenguaje natural controlado ~\cite{CNL}, para servir como lenguaje de representación de conocimiento, y de esta manera poder utilizar representaciones formales, sin necesidad de comprender la lógica interna. La ventaja de usar ACE radica en la posibilidad de generar lenguaje natural desde una representación de conocimiento en lógica, y viceversa.

Si bien estas herramientas cumplen con la verbalización de axiomas y asisten en el uso de ontologías evitando que los usuarios deban comprender las lógicas descriptivas y la sintaxis OWL, la comprensión de cada axioma aislado no asegura la comprensión del dominio completo. Un factor importante en la comprensión del dominio es la capacidad de relacionar los temas de manera coherente. La verbalización de axiomas es necesaria pero no suficiente para comprender las temáticas modeladas en la ontología. Existen conjuntos de axiomas que están relacionados y son complementarios para la descripción de entidades, y que expresados en conjunto como una unidad de información textual transmiten mejor la intención de ese conjunto de axiomas. OntoVerbal es la herramienta que tiene en cuenta este criterio, sin embargo casi no utiliza recursos lingüísticos para lograr tal objetivo, por lo que requieren otras técnicas para alcanzar una fluidez aceptable en el texto, como el uso de convenciones y plantillas, y el tratamiento de los nombres de las entidades a verbalizar. %\footnote{no se entendio esto. Habria que poner también algo asi como que se intentará organizar la información a modelar en una fase temprana de la verbalización a fin de reducir el costo que trae aparejado la manipulación de información textual en las fases ultima de la generación como lo indica las tareas propuestas por Reiter }
En NaturalOWL se tiene en cuenta la calidad del texto de salida, pero compromete la representación del dominio, agregando información lingüística y ensuciando el dominio modelado. Por último en Attempto Tools se busca un equilibrio en la calidad del texto sin comprometer el dominio modelado, pero no se hace énfasis en la comprensión del dominio, ya que únicamente se traducen los axiomas en sentencias aisladas. Por este motivo, en este trabajo se propone diseñar un algoritmo que agrupe la información relacionada a través de los temas principales presentes en la ontología. En este sentido, se propone una técnica alternativa a la de ~\cite{reiter1997building} quienes realizan todo el ordenamiento de las sentencias e información involucradas en las últimas tareas de las verbalización. Por el contrario, en nuestra propuesta se pretende organizar la mayor parte de la información antes del proceso de verbalización en sí mismo, por considerarse que en éste punto es donde se puede lograr una manipulación más fácil y limpia de la información y dejando solamente el ordenamiento mínimo en etapas posteriores. De esta manera, esperamos que el resultado de la verbalización sean sentencias en lenguaje natural relacionadas semánticamente, que comprendan unidades textuales de mayor granularidad y no únicamente unidades textuales aisladas (y posiblemente desordenadas) que cumplan una determinada sintaxis.

Con esto en mente, se propone diseñar e implementar una herramienta genérica para elegir, ordenar y generar un texto en lenguaje natural a partir de una ontología en OWL, con el fin de facilitar el entendimiento del dominio modelado, y asistir durante el proceso de modelado. % El idioma elegido para la verbalización será el inglés, principalmente por la disponibilidad de ontologías disponibles que servirán posteriormente en la etapa de validación de la herramienta.


\section{Tareas de la generación de texto}
\label{sec:tareas_gnl}
Durante el proceso de generación del texto, la entrada atraviesa diferentes etapas, creando una representación distinta en cada una. En principio, comienza como un conjunto de datos, siendo el nivel más abstracto de representación, e idealmente se avanza en la transformación atravesando distintos niveles hasta alcanzar el nivel más bajo, una representación en lenguaje natural. 

En el ámbito de la Generación de Lenguaje Natural, se engloban las tareas de cada nivel de abstracción en una etapa particular. Existen tres etapas principales: \emph{macro planning}, \emph{micro planning} y \emph{realización gramatical}.

\begin{itemize}
    \item  \emph{Macro planning:} agrupa las tareas de nivel más abstracto. Se encarga de seleccionar la información que se usará en la generación de texto y de organizarla a nivel de capítulos, secciones, párrafos, y sentencias.
    \item \emph{Micro planning:} es la etapa asociada a la construcción de sentencias, agrupa las tareas de nivel de abstracción intermedio. Dado un conjunto de datos que deben estar presentes en una sentencia, el micro planning decide en qué orden aparecerán y cómo se combinarán para producir una oración cohesiva. 
    Para esto, se selecciona las configuraciones léxicas de las palabras, se elige qué entidades se reemplazarán por una expresión de referencia, y qué oraciones se pueden combinar para mejorar la fluidez del texto.
    \item \emph{Realización gramatical y expresiones de referencia:} es la última etapa, donde se seleccionan las palabras adecuadas según lo establecido en el micro planning. Puede involucrar, por ejemplo, el uso de algoritmos de conjugación de verbos. Expresiones de referencia se refiere a la descripción de objetos.
\end{itemize}

