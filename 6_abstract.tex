\ \\
\ \\
\label{pagsumm}
\noindent{\LARGE \sc Abstract}
\\ \\
Ontologies play a key role in the formation of the Semantic Web. They allow the representation of knowledge in a formal way, giving a well-defined meaning to the data and allowing reasoners to obtain implicit information from the domain.

The Semantic Web is defined as an extension of the current Web. It aims to explicitly add a layer of meaning to the data, in order to create a better environment for automating complex tasks. It is at this point that ontologies are useful, providing the appropriate tools to structure the Semantic Web.

Taking into account that ontologies structure data in a formal way, they are poorly understood by non-expert users, and do not provide a comfortable display of information for those who want to benefit from the use of semantic technologies.

Although there are some applications for developing and exploring ontologies, they are not satisfactory enough for non-expert or end users.
For this reason, expressing formal content in Natural Language (LN) is attractive, providing the ability to document and express ontologies in a language accessible by users not trained in mathematics or the modeled domain.


However, expressing the formal content in LN is not useful enough if only isolated sentences are generated. For a text to be beneficial to an end user, it must be organized and consistent. In this way the relationships between the concepts can be captured, to understand the complete domain and not only interpret its isolated axioms. For this reason, a Natural Language Generation System (SGLN) was designed and developed, which allows an organized text to be obtained from the content of an ontology in OWL 2 language.


The result of this work is a system that requires a minimum commitment in the design of ontologies, and produces a syntactically acceptable and organized output.

The two key points in this work are: the \textit{Organization of Information}, to structure the text based on semantic relationships, in order to reduce the cognitive load that requires recognizing domain relationships; and the \textit{Natural Language Text Generation}, maximizing the cohesion of sentences.


\vfill
\pagebreak
