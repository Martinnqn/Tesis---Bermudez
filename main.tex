\documentclass[11pt,a4paper,spanish]{book}
\usepackage[a4paper,left=3cm,right=2cm,top=3cm,bottom=2cm]{geometry}
%\usepackage[margin=1in]{geometry}
\usepackage[utf8]{inputenc}
\usepackage[spanish, es-tabla]{babel}
\usepackage[]{enumerate}
\usepackage[]{amssymb}
\usepackage[colorlinks, citecolor=black, filecolor=black, linkcolor=black, urlcolor=black]{hyperref}
\usepackage{graphicx}
\usepackage{float}
\usepackage{multirow}
\usepackage{subfigure}
\usepackage{caption}
\usepackage{minted}
\usepackage[table,xcdraw]{xcolor}
\usepackage{tabularx,ragged2e}
\usepackage{textcomp}
\usepackage{multicol}
\usepackage{tikz}
\usepackage{adjustbox}
%para gramatica bnf
\usepackage{bnf}
%para definir nuevo environment
\usepackage{newfloat}
%para insertar pdf
\usepackage[final]{pdfpages}
%para el arbol de directorios
\usepackage{dirtree}
\usepackage{verbatim} 

\usepackage{algorithm}
\usepackage{algorithmic}

%para el caption de las listas
\DeclareCaptionType{listCap}[Lista]


% declare the floating environment {Grammar}
% this will also define \listofGrammars:
\DeclareFloatingEnvironment[
  % the file extension for the file used to create the list:
  fileext   = logr,% don't use log here!
  % the heading for the list:
  listname  = {List of Grammars},
  % the name used in captions:
  name      = Gramática,
  % the default floating parameters if the environment is used
  % without optional argument:
  placement = htp
]{GrammarEnv}

\newtheorem{definition}{{\bf\sc Definición }}[section]
\newtheorem{theorem}{Teorema}[section]

\newcolumntype{C}{>{\Centering\arraybackslash}X} % centered "X" column

%separacion de columnas multicol
 \setlength{\columnsep}{5mm}

\floatname{algorithm}{Algoritmo}
\renewcommand{\algorithmicrequire}{\textbf{Input:}}
\renewcommand{\algorithmicensure}{\textbf{Output:}}


\begin{document}
	
\title{Generación de texto en la web semántica}
\author{\ }
\date{}
\maketitle

%Empieza la numeración en números romanos
\frontmatter
\tableofcontents
%Empieza la numeración en números arábigos
\mainmatter

\chapter{Introducción}
La Web Semántica es un conjunto de estándares y tecnologías. Tiene como objetivo enriquecer la web actual (Web 2.0) agregando explícitamente una capa de significado a los datos. De esta manera, se crea un mejor ambiente para poder automatizar tareas complejas que actualmente se realizan por los humanos de forma manual. 

Para describir el dominio de interés en la Web Semántica, se adoptó el uso de ontologías, que tienen base en la lógica descriptiva y el vocabulario de RDF.

Como las ontologías estructuran los datos de manera formal, son poco comprensibles por usuarios no expertos, y no brindan una cómoda visualización de la información para aquellos que quieran beneficiarse del uso de las tecnologías semánticas. Por este motivo, expresar el contenido formal en Lenguaje Natural (LN) resulta atractivo, brindando la capacidad de documentar y expresar ontologías en un lenguaje accesible por usuarios no entrenados en matemáticas o el dominio modelado. 

Por eso en esta tesis se propone un sistema de generación de textos, capaz de generar un documento de texto organizado, a partir del contenido de una ontología expresada en Lenguaje OWL.

\section{Objetivo}\label{Intro:objetivo}
El objetivo de este trabajo es diseñar e implementar una aplicación capaz de procesar el contenido de una ontología y expresarlo en lenguaje natural, para que sea accesible a cualquier usuario cuyo interés sea el dominio modelado, pero no tenga las herramientas necesarias para comprender las lógicas subyacentes.

Se espera que este trabajo sirva como soporte a la comprensión de los dominios modelados en ontologías.


\section{Motivación}
La motivación principal es facilitar la comprensión de ontologías. Utilizar procesos de generación de texto automática resultan convenientes para generar documentos de texto que sean comprensibles por cualquier usuario.

Por otro lado, se encuentra el hecho de acercar más las disciplinas tales como Lingüística Teórica y Ciencias de la Computación. El papel que ha desempeñado la Lingüística Teórica en el Procesamiento de Lenguaje Natural ha sido poco notorio~\cite{perinan2012defensa}.



\chapter{Antecedentes conceptuales}

\section{Trabajos relacionados}
%%copiado y pegado de la propuesta

El uso de ontologías para la descripción de dominios se ha ido intensificando en los últimos años. La utilidad de las ontologías radica en la capacidad de compartir e integrar datos, y del potencial de inferir conocimiento a través de agentes informáticos. Sin embargo, el desarrollo y análisis de ontologías resulta muy difícil para los seres humanos, tanto para usuarios expertos como para los que no son expertos en ontologías, por lo que es imprescindible el uso de herramientas de soporte para facilitar el trabajo con las mismas. Actualmente existen algunas aplicaciones para desarrollar y explorar ontologías, como Protégé~\cite{protege}, Mindswap~\cite{golbeck2002new}, OntoEdit~\cite{sure2002ontoedit}, OntoLingua \cite{farquhar1997ontolingua}, WebOnto~\cite{domingue1998tadzebao}, entre otras, pero no resultan suficientemente satisfactorias para usuarios no expertos, o usuarios finales. 

Varios trabajos involucran el procesamiento de lenguaje natural en el ámbito de las ontologías. Por un lado, existen herramientas que se enfocan en la verbalización de un conjunto de axiomas pertenecientes a una ontología, con el fin de generar un conjunto de sentencias que puedan ser utilizadas para reconocer la información perteneciente al dominio modelado, o documentar la ontología, utilizando lenguaje natural. En este tipo de trabajos se encuentran algunas herramientas como OntoVerbal~\cite{liang2013ontoverbal}, que utiliza la frecuencia con la que se utilizan conjuntos de axiomas para la descripción de entidades, y describe patrones lingüísticos para verbalizar los conjuntos de mayor frecuencia. Además, agrupa los axiomas alrededor de un tópico para crear párrafos. 
Además, genera la creación de párrafos mediante la agrupación de axiomas alrededor de un tópico dado.
Otra herramienta es NaturalOWL~\cite{galanis2007generating}, que se enfoca en la flexibilidad y fluidez del texto, agregando información lingüística a la ontología para verbalizar los axiomas.

Por otro lado, se han desarrollado herramientas para asistir durante el desarrollo de la ontología, permitiendo acceder a la información modelada en lógicas descriptivas en un formato de texto, y poder editarlo en lenguaje natural. Entre las herramientas desarrolladas se encuentran CLOnE~\cite{power2010complexity} y Aqualog~\cite{lopez2005aqualog},completamente orientadas a la comprensión de texto y no a la generación; y Attempto Tools~\cite{attempto}, que utiliza Attempto Controlled English  (ACE) que es un lenguaje natural controlado ~\cite{CNL}, para servir como lenguaje de representación de conocimiento, y de esta manera poder utilizar representaciones formales, sin necesidad de comprender la lógica interna. La ventaja de usar ACE radica en la posibilidad de generar lenguaje natural desde una representación de conocimiento en lógica, y viceversa.

Si bien estas herramientas cumplen con la verbalización de axiomas y asisten en el uso de ontologías evitando que los usuarios deban comprender las lógicas descriptivas y la sintaxis OWL, la comprensión de cada axioma aislado no asegura la comprensión del dominio completo. Un factor importante en la comprensión del dominio es la capacidad de relacionar los temas de manera coherente. La verbalización de axiomas es necesaria pero no suficiente para comprender las temáticas modeladas en la ontología. Existen conjuntos de axiomas que están relacionados y son complementarios para la descripción de entidades, y que expresados en conjunto como una unidad de información textual transmiten mejor la intención de ese conjunto de axiomas. OntoVerbal es la herramienta que tiene en cuenta este criterio, sin embargo casi no utiliza recursos lingüísticos para lograr tal objetivo, por lo que requieren otras técnicas para alcanzar una fluidez aceptable en el texto, como el uso de convenciones y plantillas, y el tratamiento de los nombres de las entidades a verbalizar. 
En NaturalOWL se tiene en cuenta la calidad del texto de salida, pero compromete la representación del dominio, agregando información lingüística y ensuciando el dominio modelado. Por último en Attempto Tools se busca un equilibrio en la calidad del texto sin comprometer el dominio modelado, pero no se hace énfasis en la comprensión del dominio, ya que únicamente se traducen los axiomas en sentencias aisladas. Por este motivo, en este trabajo se propone diseñar un algoritmo que agrupe la información relacionada a través de los temas principales presentes en la ontología. En este sentido, se propone una técnica alternativa a la de ~\cite{reiter1997building} quienes realizan todo el ordenamiento de las sentencias e información involucradas en las últimas tareas de las verbalización. Por el contrario, en nuestra propuesta se pretende organizar la mayor parte de la información antes del proceso de verbalización en sí mismo, por considerarse que en éste punto es donde se puede lograr una manipulación más fácil y limpia de la información y dejando solamente el ordenamiento mínimo en etapas posteriores. De esta manera, esperamos que el resultado de la verbalización sean sentencias en lenguaje natural relacionadas semánticamente, que comprendan unidades textuales de mayor granularidad y no únicamente unidades textuales  aisladas (y posiblemente desordenadas)a nivel de oración  que cumplan una determinada sintaxis.

Con esto en mente, se propone diseñar e implementar una herramienta genérica para elegir, ordenar y generar un texto en lenguaje natural a partir de una ontología en OWL, con el fin de facilitar el entendimiento del dominio modelado, y asistir durante el proceso de modelado. % El idioma elegido para la verbalización será el inglés, principalmente por la disponibilidad de ontologías disponibles que servirán posteriormente en la etapa de validación de la herramienta.


\section{Tareas de la generación de texto}
\label{sec:tareas_gnl}
Durante el proceso de generación del texto, la entrada atraviesa diferentes etapas, creando una representación distinta en cada una. En principio, comienza como un conjunto de datos, siendo el nivel más abstracto de representación, e idealmente se avanza en la transformación atravesando distintos niveles hasta alcanzar el nivel más bajo, una representación en lenguaje natural. 

En el ámbito de la Generación de Lenguaje Natural, se engloban las tareas de cada nivel de abstracción en una etapa particular. Existen tres etapas principales: \emph{macro planning}, \emph{micro planning} y \emph{realización gramatical}.

\begin{itemize}
    \item  \emph{Macro planning:} agrupa las tareas de nivel más abstracto. Se encarga de seleccionar la información que se usará en la generación de texto y de organizarla a nivel de capítulos, secciones, párrafos, y sentencias.
    \item \emph{Micro planning:} es la etapa asociada a la construcción de sentencias, agrupa las tareas de nivel de abstracción intermedio. Dado un conjunto de datos que deben estar presentes en una sentencia, el \emph{micro planning} decide en qué orden aparecerán y cómo se combinarán para producir una oración cohesiva. 
    Para esto, se selecciona las configuraciones léxicas de las palabras, se elige qué entidades se reemplazarán por una expresión de referencia, y qué oraciones se pueden combinar para mejorar la fluidez del texto.
    \item \emph{Realización gramatical y expresiones de referencia:} es la última etapa, donde se seleccionan las palabras adecuadas según lo establecido en el \emph{micro planning}. Puede involucrar, por ejemplo, el uso de algoritmos de conjugación de verbos. Expresiones de referencia se refiere a la descripción de objetos.
\end{itemize}

\section{Teoría de grafos}
La teoría de grafos se encarga de estudiar las propiedades de los grafos. Un grafo es una estructura de datos que consiste de un conjunto de vértices conectados por un conjunto de arcos que pueden ser usados para modelar relaciones entre los objetos en una colección~\cite{mihalcea2011graph}.

Algunos aspectos que se estudian de los grafos son su estructura (jerárquica, árbol, conexos, bipartitos, etc.), las medidas de sus nodos (centralidad relativa, absoluta~\cite{freeman1978centrality}), los algoritmos de búsqueda (profundidad, anchura, etc.), etc.

La teoría de grafos tiene diversas aplicaciones, por ejemplo el análisis de redes sociales, representación de mapas conceptuales, diseño de circuitos eléctricos, topologías de redes de computadoras, etc. Actualmente se estudia la teoría de grafos y el procesamiento de lenguaje natural en conjunto, para hallar solución a diferentes problemas del procesamiento de texto, como el etiquetado léxico, clasificación de texto y clustering~\cite{mihalcea2011graph}, desambiguación semántica de palabras~\cite{MihalceaSinhaDisambiguation}\cite{ArabJahromiDisambiguation}, extracción de keywords~\cite{Litvak:2008:GKE:1613172.1613178}\cite{beliga2015overview}, generación de resúmenes~\cite{plaza2011uso}, entre otras.

\subsection{Criterio de centralidad de un nodo}
El criterio de centralidad es alguna función del grado de un nodo~\cite{freeman1978centrality}. El grado de un nodo $n$ es la cuenta de la cantidad de nodos adyacentes a $n$. 

La maedida de centralidad es utilizada para cuantificar la importancia de un nodo~\cite{2018transfer}. Un ejemplo del cálculo de centralidad es \emph{PageRank}~\cite{page1999pagerank} el algoritmo utilizado por Google para calcular la relevancia de los documentos web.

Existen diferentes medidas de centralidad, algunas de ellas son:
\begin{itemize}
    \item \emph{Degree Centrality}: en grafos no dirigidos es la cuenta de la cantidad de conexiones de un nodo; en un grafo dirigido, se tiene en cuenta la dirección de la conexión, siendo outdegree la medida de la cantidad de conexiones de salida de un nodo, e indegree la medida de la cantidad de conexiones entrantes al nodo. 
    \item \emph{Closeness Centrality}: es la longitud promedio del camino más corto entre el nodo y todos los demás nodos en el grafo.
    \item \emph{Between Centrality}: calcula la cantidad de veces que un nodo actúa como un puente en el camino más corto entre otros dos nodos.
\end{itemize}

\subsection{Algoritmos de recorridos de grafos}
Se utilizan para recorrer los nodos de un grafo. Dos clasificaciones principales:
\begin{itemize}
    \item Recorrido en profundidad: dado un nodo $n$, se visita un nodo $n'$ adyacente, explorando cada camino que salga de él. Hasta que no se haya finalizado de explorar uno de los caminos no se comienza con el siguiente. Un camino deja de explorarse cuando se llega a un nodo ya visitado. 
    \item Recorrido en anchura: dado un nodo $n$, se visitan todos sus nodos adyacentes. Sea $n'$ un adyacente de $n$, no se visita ningún adyacente de $n'$ hasta que ya se hayan visitado todos los adyacentes de $n$. 
\end{itemize}

\subsection{Procesamiento de Lenguaje Natural}
La representación de texto como una estructura de datos basada en grafos ha sido usada para dar soporte a las tareas de procesamiento de lenguaje natural. En TextGraph Workshop Series\footnote{http://www.textgraphs.org} se pone a disposición una plataforma para el intercambio de ideas que beneficien ambos campos de investigación. 

La representación de datos como un grafo beneficia el PNL poniendo a disposición los algoritmos conocidos, permitiendo la navegación del grafo para extraer y clasificar el contenido. Algunas tareas de PLN que son resueltas utilizando grafos son la creación de resúmenes~\cite{mihalcea2004graph}\cite{erkan2004lexrank} y minería de texto~\cite{Jin:2007:GTR:1244002.1244182}\cite{beliga2015overview}.

\begin{itemize}
    \item Resúmenes: ``La generación automática de resúmenes consiste en la creación de una versión reducida de uno o varios documentos por parte de un programa de ordenador, de tal forma que el resumen producido condense la información importante del texto de entrada''~\cite{plaza2011uso}.
    \item Minería de texto: ``el proceso de extracción automática de información fundamental de textos,  detección  automática  de temas  predominantes en un conjunto de documentos y búsqueda de textos relevantes mediante consultas de grandes prestaciones y flexibilidad''~\cite{brun2004articulo}.
\end{itemize}

\subsection{Representación de Ontologías}
Tanto en la academia como en la industria~\cite{soylu2018navigating} se han utilizado grafos para la representación de conocimiento. Este enfoque permite explotar las propiedades de un grafo dentro del área de la visualización de la información, por ejemplo para la representación y navegación de ontologías~\cite{escarza2005visualizacion}.

El uso de herencia múltiple en ontologías genera estructuras de datos en forma de grafos dirigidos, sin embargo se pueden usar diferentes criterios para mostrar el contenido de una ontología, por ejemplo Protégé utiliza una jerarquía en forma de árbol para mostrar la jerarquía de clases, donde cada nodo del árbol es una clase\footnote{citar http://protegeproject.github.io/protege/views/class-hierarchy/}. Como es posible que un nodo sea hijo de más de un padre, existe redundancia en la visualización, ya que un nodo puede repetirse en dos ramas diferentes.

En~\cite{mellish2008natural} presentan una investigación acerca de \emph{Natural Language Directed Inference from Ontologies} (NLDI), en el cual buscan determinar el contenido para describir una entidad presente en una ontología. Los axiomas elegidos para tal descripción surgen de aplicar un algoritmo \emph{best-fist search} sobre un grafo generado a partir de los axiomas de la ontología.

En~\cite{crampes2007ontology} comparan dos aplicaciones que utilizan ontologías para soportar y dirigir la navegación conceptual. En ellas, aplican criterios de selección y ordenamientos de conceptos a partir de las relaciones de las entidades en la ontología del dominio y de reglas específicas de una ontología pedagógica. 

De estos trabajos se observan dos aspectos importantes para integrar la teoría de grafos y la representación de las ontologías:
\begin{itemize}
    \item Determinar qué información de la ontología será usada para crear los nodos y los enlaces del grafo.
    \item Establecer un método de navegación del grafo.
\end{itemize}

Dado que se trabaja con grafos semánticos, es deseable aprovechar al máximo la semántica de las relaciones, ya que cada ruta entre los axiomas corresponden a diferentes transiciones posibles en un texto coherente~\cite{mellish2008natural}.


\chapter{Presentación del problema}

Para cumplir con el objetivo propuesto, se dividirá el desarrollo del sistema de generación de texto en dos problemas. El primero consiste en crear un módulo que preprocese la entrada para organizar y estructurar la información de forma más significativa. Reiter E. ha propuesto esta etapa previa de preprocesamiento~\cite{reiter2007architecture} para el análisis y la interpretación de datos, aunque fue aplicada en una entrada de datos en bruto en lugar de una base de conocimiento.

El segundo problema consiste en diseñar e implementar el módulo que toma como entrada la información organizada y la convierte en un documento de texto. Para esto usaremos como referencia las tareas descritas en la Sección~\ref{sec:tareas_gnl}.

Se puede apreciar la composición de nuestro sistema de generación de lenguaje natural en la Figura~\ref{fig:modulos_sgln}.

\begin{figure}
    \centering
    \includegraphics[width=12cm, height=4cm]{img/presentacion_problema/modulos_sgln.pdf}
    \caption{Módulos que componen el sistema de generación de lenguaje natural}
    \label{fig:modulos_sgln}
\end{figure}

\section{Presentación del problema de la organización de la información para la coherencia del texto}
\label{sec:problema_coherencia-texto}
 En la generación de texto automática, además de desear cumplir ciertas características léxicas y sintácticas del lenguaje destino, se requiere que las oraciones del texto posean una conexión semántica, para dar pie a la coherencia del texto. 
 %%%
Si bien existen diferentes formas de organizar la información para formar un texto coherente, debemos exigir que entre las unidades de comunicación adyacentes, haya relación entre los tópicos que presentan, y no únicamente una relación semántica. Por ejemplo, el siguiente fragmento de texto fue presentado en \cite{van1983ciencia}:
``Compré esta máquina de escribir en Nueva York. Nueva York es una gran ciudad de USA. Las grandes ciudades a veces tienen serios problemas financieros''. A medida que se avanza en el texto, se presenta una conexión lineal entre algunos componentes de cada oración, pero no parece transmitir un mensaje claro, pues, los tópicos de cada oración son distintos. De igual manera sucede entre párrafos y capítulos que no mantienen un tópico en común. 

Este salto entre diferentes temas dificulta la interpretación del texto. Considerando el ámbito de las ontologías, resulta conveniente mantener la centralidad del tema mientras se describe el dominio, para que el lector pueda hacer las conexiones adecuadas entre las entidades del dominio.

Para encarar esta problemática, existen diferentes investigaciones que buscan organizar sentencias midiendo la fuerza de sus relaciones [referenciar a alguna de las investigaciones en macroplanning]. 
\\

En una ontología, la información se representa en forma de entidades relacionadas con otras entidades. Una manera de representar visualmente un conjunto de entidades, sus relaciones y atributos es mediante el uso de grafos \cite{escarza2005visualizacion}. Podemos valernos de la estructura del grafo, para medir el grado de relación entre las entidades, antes del proceso de verbalización. Se pueden utilizar diferentes criterios para medir el grado de relación. A continuación veremos un ejemplo a partir de la Figura~\ref{fig:pizza.owl}, la cual contiene representada gráficamente un subconjunto de la ontología pizza.owl\footnote{https://protege.stanford.edu/ontologies/pizza/pizza.owl}.

\begin{figure}
\centering
\subfigure[Jerarquía de clases de la ontología pizza]{\includegraphics[width=70mm]{img/presentacion_problema/onto_pizza.pdf}}
\subfigure[Jerarquía de propiedades de la ontología pizza]{\includegraphics[width=3cm]{img/presentacion_problema/onto_pizza_properties.pdf}}
\caption{Subconjunto de la ontología de pizzas.} \label{fig:pizza.owl}
\end{figure}

Aún antes de generar las proposiciones que reflejan la información del grafo, se puede ver que hay formas más adecuadas que otras para comenzar a transmitirle a un receptor toda la información del grafo. Algunos de los posibles tópicos que engloban o relacionan a la mayoría de los datos expuestos son ``Tipo de comidas'', o ``Ingredientes de la pizza'', siendo que, la mayor cantidad de información está relacionada a la pizza. 

Para identificar los nodos más relevantes, pueden usarse diferentes criterios sobre la estructura del grafo, tal como la disposición jerárquica, o la cantidad de relaciones (el grado) de cada nodo. También se puede hacer uso de la jerarquía de las propiedades, para reconocer su tipo más general, como en \emph{tiene cobertura} y \emph{tiene base}, con el fin de referirse a ellas como ``Ingredientes''.

Para mostrar un ejemplo, supondremos una ontología que represente el texto de la compra de la máquina de escribir. Puede verse el grafo de la ontología en la Figura~\ref{fig:maquina_escr}.

\begin{figure}
    \centering
    \includegraphics[scale=0.65]{img/presentacion_problema/onto_maq_escr.pdf}
    \caption{Ontología del texto de la máquina de escribir.}
    \label{fig:maquina_escr}
\end{figure}

Basándonos en el texto propuesto, puede verse que para reproducirlo a partir de la ontología propuesta, se debe recorrer el grafo en profundidad, comenzando desde la clase \emph{Persona}. De esta manera, aunque los datos tengan una relación semántica, en la generación del texto no queda claro el objetivo o la finalidad del texto. 

Si en lugar de recorrer el grafo en profundidad, se recorre en anchura, se produce un texto con más sentido, ya que centra como tópico principal a la máquina de escribir. Por ejemplo, si comenzamos desde \emph{Persona}, podemos formar el texto ``Compré una máquina de escribir. Es de marca X y modelo Y, con color negro. Fue fabricada en New York''.


En estos ejemplos hemos presentado dos problemáticas a tratar: la obtención de los tópicos más relevantes para tratar en el texto, que en nuestro caso son unidades de información representadas por \emph{owl:Class} y \emph{owl:NamedIndividual}; y el recorrido del grafo para obtener las unidades de información adecuada que maximicen la relación semántica respecto a los tópicos del texto. En este caso, las unidades de información que se tienen en cuenta para el recorrido del grafo son ambas \emph{owl:Class} y \emph{owl:ObjectProperty}.
 
\section{Presentación del problema de verbalización}
%Como presentamos en \ref{sec:tareas_gnl}, existen ciertas tareas a desarrollar en un sistema de generación de texto.
Como presentamos en el Objetivo, se esperan alcanzar dos cualidades principales en el texto de salida:
\begin{enumerate}
    \item A nivel macro, el texto debe tener una estructura que permita visualizar los principales temas, presentándolos de manera aislada (secciones, párrafos), pero que en conjunto se complementen para que el lector pueda establecer las relaciones adecuadas entre ellos. Como objetivo principal se espera alcanzar una estructura que presente la información más importante y necesaria cuanto antes, preparando al lector para que pueda comprender información nueva.
    \item A nivel micro, el texto debe cumplir con cierta fluidez, utilizando frases que tengan una gramática aceptable, tratando de maximizar la cohesión.
\end{enumerate}{}

\chapter{Organización de la información}

\section{Introducción}
Para abordar los problemas de coherencia propuestos en la Sección~\ref{sec:problema_coherencia-texto}, es necesario crear una estructura jerárquica para guiar el desarrollo del texto. Esta estructura está compuesta por las entidades presentes en la ontología, de manera que las entidades que consideramos más relevantes se encuentren en el primer nivel de jerarquía, y a medida que se descienda por los niveles la importancia de las entidades disminuya.

\section{Diseño}
Para alcanzar esta estructura, se crearon cuatro módulos que transforman la ontología de entrada en un árbol con toda la información contenida en la misma.
A continuación se describe cada módulo:
\begin{itemize}
    \item Traductor: recibe la entrada y la traduce a una representación interna de la ontología.
    \item Clasificador de Entidades: clasifica y extrae las entidades más relevantes, utilizando como criterio las medidas de centralidad.
    \item Generador de Nuevo Grupo de Entidades: recibe una Entidad y recorre sus relaciones para crear un nuevo grupo de entidades relacionados semánticamente.
    \item Generador del Árbol de Entidades: se encarga de crear los niveles de la estructura jerárquica, interactuando con el Generador de Nuevo Grupo de Entidades.
    
\end{itemize}
%Para finalizar, un último módulo agrega en una nueva rama del árbol a aquellas entidades que no han sido utilizadas en ninguna otra rama.

En la Figura~\ref{fig:modulos_organizador_inf} se muestra la transformación de la ontología OWL a través de los módulos. Los pasos 4 y 5 son iterativos, el proceso finaliza cuando no queden más Entidades que explorar, lo cual es definido por el Generador del Árbol de Entidades.

\begin{figure}[H]
    \centering
    \includegraphics%[width=12cm, height=6cm]
    [scale=1]{img/organizacion_informacion/modulos_organizador_de_informacion.pdf}
    \caption{Módulos que componen el organizador de información}
    \label{fig:modulos_organizador_inf}
\end{figure}


\subsection{Traductor de la entrada}
El módulo que crea la representación interna de la ontología se encarga de cargar toda la información de la ontología para poder ser accedida con mayor facilidad. Como algunos datos son accedidos a través de un razonador o importados de otras ontologías, se optó por cargar toda esta información una única vez para reducir el tiempo de espera cuando se requieren estos datos.

La estructura captura los componentes de la ontología, y los relaciona para poder ser accedidos y recorridos como un grafo. 

\subsection{Clasificación de las entidades más relevantes}
Una vez creada la representación de la ontología, el segundo módulo debe clasificar las entidades y extraer las más relevantes. A continuación se explicará cómo son extraídas estas entidades.

\subsubsection{Criterio de clasificación}
El criterio principal para establecer una clasificación sobre las clases de una ontología, y que tal clasificación sirva para seleccionar las clases más relevantes, se basa en cuánta información posee una clase respecto al resto de las clases en la ontología. Consideramos que una clase tiene más información que otra si está presente en más cantidad de dominios de \emph{ObjectProperties}. Esto permite reconocer las clases que tengan mayor cantidad de conexiones con otras clases, y que a su vez sean el núcleo de la relación. 

Para obtener la cantidad de información de cada clase, utilizaremos el grafo subyacente a la ontología y calcularemos el \emph{outdegree} de cada nodo que represente una clase. Para el cálculo se tendrá en cuenta únicamente la relación \emph{rdf:domain}.

Teniendo en cuenta a las clases mejor clasificadas, se puede centrar lo que se dice en el texto alrededor de estas clases.

Algunas ventajas de este enfoque son:
\begin{itemize}
    \item Sencillez de implementación. Únicamente se debe recorrer el grafo calculando el valor de cada nodo. El recorrido del grafo tiene a lo sumo una complejidad polinomial.
    \item No se requiere agregar información extra al dominio.
    \item No es necesario utilizar sobre la ontología un razonador que requiera una complejidad computacional que sea intratable. La única observación es que hay que inferir el dominio y rango de las \emph{ObjectProperties}. Sin embargo, la inferencia de dominio y rango se realiza teniendo en cuenta la jerarquía de \emph{ObjectProperties}, sin necesidad de inferir clases equivalentes o disjuntas.
\end{itemize}

La eficacia de este enfoque depende fuertemente de que las \emph{ObjectProperties} más importantes tengan el dominio explícito.


\subsubsection{Seleccionando las principales clases}
\label{sec:select_class}
Una vez calculado el valor de cada clase, se pueden seleccionar las clases que superen cierto umbral para agregarlas al primer nivel de la estructura jerárquica que guiará el desarrollo del texto. En este trabajo, se recorren todas las clases y se seleccionan las que superen el valor promedio entre cantidad de propiedades y cantidad de clases.

Si ninguna supera el valor promedio, se selecciona la o las clases con el valor más alto.

Cuando se seleccionan las clases que van a representar los temas principales, puede ocurrir que se elijan clases de una misma rama en la jerarquía de clases. Esto hace que se pierda algo de semántica, pues se incluyen clases en el mismo nivel, siendo que algunas son más específicas y pueden ser alcanzadas desde sus ancestros. Para evitar este problema, se optó por eliminar las subclases que sean seleccionadas en un principio, y que tengan a una clase ancestro dentro de los temas principales. Por ejemplo, continuando con la ontología de las pizzas, si las principales clases seleccionadas son \emph{comida} y \emph{pizza}, la clase \emph{pizza} sería eliminada del conjunto, pues es subclase de \emph{comida}. En las secciones siguientes se verá que la subclase \emph{pizza} aparecerá en el siguiente nivel correspondiente a la rama \emph{comida}.

\subsection{Agrupando la información de las clases elegidas}
\label{sec:agrupando_info}
Una vez elegidas las clases que estarán en el primer nivel de la jerarquía, se procede a crear los niveles siguientes, con toda la información relacionada a estas clases. Para esto se creó un tercer módulo, que agrupa la información de cada entidad en grupos generales, que abarquen con el mayor nivel de abstracción posible su información. Para cada clase, se busca reemplazar las propiedades donde participan como dominio, por propiedades en común de más alto nivel en la jerarquía de \emph{ObjectProperties}. Por ejemplo, en la ontología de las pizzas, para la clase \emph{pizza} existen dos propiedades: \emph{tieneCobertura} y \emph{tieneBase}. Como estas dos propiedades tienen a su vez una \emph{superproperty} en común llamada \emph{tieneIngrediente}, se agrupa toda la información que pueda ser alcanzada por las \emph{subproperties} en un único grupo representado por \emph{tieneIngrediente}. 

Además de la información que puede ser alcanzada desde las \emph{ObjectProperties}, la información que poseen las Subclases y los Individuos también es agrupada. En este caso estamos haciendo uso de las relaciones \emph{owl:subClassOf} y \emph{owl:NamedIndividual}.

En la Figura~\ref{fig:diagrama_secuencia_contentGrouping} se muestra un pseudocódigo del algoritmo usado para agrupar la información de las clases.

Con toda esta información, se crea un primer nivel de grupos que abarcan lo más general posible todos los temas a ser tratados en un texto. Como ejemplo, el primer nivel de la ontología de las pizzas quedaría como en la figura~\ref{fig:macro_planning_pizza}.

\begin{figure}[H]
\centering
\begin{minipage}[c]{0.7\textwidth}
%no borrar el % de dirtree porque es necesario.
{\footnotesize 
\dirtree{%
.1 pizza.
.2 tieneIngrediente.
.2 pizzaConCarne.
.2 pizzaConNombre.
.2 $\ldots$ las restantes subclases de pizza.
}}
\caption{Organización del contenido de la ontología \emph{pizza}.}
\label{fig:macro_planning_pizza}
\end{minipage}
\end{figure}


\begin{figure}[H]
    \centering
    \includegraphics%[width=8cm, height=7cm]
    [scale=0.8]{img/organizacion_informacion/secuencia_contentGrouping}
    \caption{Diagrama de secuencia para agrupar la información de las clases}
    \label{fig:diagrama_secuencia_contentGrouping}
\end{figure}


\subsubsection{Recorriendo la ontología}
Luego de obtener el primer grupo de información, queda pendiente tratar el segundo problema nombrado en la Sección~\ref{sec:problema_coherencia-texto}: recorrer la ontología obteniendo y agrupando la información relacionada a los temas elegidos. 

El objetivo es establecer una jerarquía de grupos de información, comenzando desde los elementos del primer grupo, que abarcan la información más general, e ir creando nuevos grupos con información más específicos. De esta manera, se garantiza que entre los elementos pertenecientes a una rama haya una relación semántica.

Partiendo desde el primer nivel (como los de la Figura~\ref{fig:macro_planning_pizza}), se aplica el mismo algoritmo usado en la Sección~\ref{sec:agrupando_info} para obtener el contenido principal, pero esta vez sobre cada elemento de dicho nivel. Como ahora es posible que un elemento sea una \emph{ObjectProperty}, para continuar el recorrido de la ontología se buscan sus {\tt subproperties}, para luego continuar obteniendo información de ellas. Las subproperties elegidas deben contener en su dominio, a la clase que haya sido elegida como tópico para ese grupo de información que se está creando. La búsqueda de subproperties debe ser exhaustiva, pues si no se encuentran subproperties que tengan en su dominio a la clase tópico, debe continuar la búsqueda bajando en la jerarquía de propiedades hasta hallar alguna, ya que es el proceso inverso a la búsqueda de superproperties en común usadas para agrupar la información de manera más general. Si no hay subproperties, entonces la nueva información para agregar en el texto se basa en el rango de la \emph{ObjectProperty} inicial. En este último caso, empleamos la relación \emph{rdf:Range}.

Continuando con el ejemplo de la ontología de la pizza, expandimos la figura~\ref{fig:macro_planning_pizza} con el siguiente nivel de información, en la figura~\ref{fig:macro_planning_pizza_n2}. 
\begin{figure}[H]
\centering
\begin{minipage}[c]{0.7\textwidth}
%no borrar el % de dirtree porque es necesario.
{\footnotesize 
\dirtree{%
.1 pizza.
.2 tieneIngrediente.
.3 tieneBase.
.3 tieneCobertura.
.2 pizzaConCarne.
.2 pizzaConNombre.
.3 Margherita.
.3 Napoletana.
.3 $...$ las restantes subclases de pizzaConNombre.
.2 $...$ las restantes subclases de pizza, con sus respectivas subclases.
}}
\caption{Organización del contenido de nivel dos de la ontología \emph{pizza}.}
\label{fig:macro_planning_pizza_n2}
\end{minipage}
\end{figure}

Se puede ver que las subproperties \emph{tieneBase} y \emph{tieneCobertura} tienen en su dominio a \emph{pizza}, la clase más cercana recorriendo sus ancestros. Si existiera por ejemplo, \emph{tieneCondimento} como una tercer subproperty de \emph{tieneIngrediente}, cuyo dominio no tuviera \emph{pizza}, entonces no sería listada dentro del grupo \emph{tieneIngredientes} en la rama de \emph{pizza}. Sin embargo, deben explorarse las subproperties de \emph{tieneCondimento}, por el caso de que tenga alguna subproperty que tuviera como dominio a \emph{pizza}. Por ejemplo, supongamos que \emph{tieneCondimento} tiene como subproperty a \emph{tieneOrégano} con diminio \emph{pizza}, en ese caso se habilita a \emph{tieneCondimento} para ser insertado en el grupo junto a  \emph{tieneCobertura} y \emph{tieneBase}.  

De esta manera, el mismo algoritmo se aplica recursivamente para cada elemento de cada nivel. El algoritmo se detiene cuando no encuentra más subclases ni propiedades para cada clase del último nivel.

\subsection{Agregando información marginada}
Con el recorrido de la ontología para formar los grupos de información, existe la posibilidad de que hayan clases que no sean parte de un grupo. Es el caso, por ejemplo, de la clase \emph{Comida}, y sus subclases \emph{Helado} y \emph{Condimento}, en la ontología \emph{pizza}. Para no dejar información del dominio excluida en el árbol, se ejecuta un cuarto módulo, que crea un nuevo grupo menos importante, y se agregan todas las clases que habían quedado marginadas. El proceso de creación de subgrupos para este nuevo grupo es idéntico al explicado en las sección anterior.

Continuando con el ejemplo de la pizza, el resultado de agregar la  información marginada se muestra en la figura~\ref{fig:macro_planning_pizza_marg}

\begin{figure}[H]
\centering
\begin{minipage}[c]{0.7\textwidth}
{\footnotesize 
\dirtree{%
.1 pizza.
.2 ....
.1 comida.
.2 condimento.
.2 helado.
}
}
\caption{Grupos de la ontología \emph{pizza} con información marginada.}
\label{fig:macro_planning_pizza_marg}
\end{minipage}
\end{figure}

\subsection{Diagrama de clases}
En la Figura~\ref{fig:diagrama_clases_organizador} se muestra el diagrama de clases correspondiente al Organizador de Información. El módulo \emph{Traductor} está representado a través de la clase \emph{OntologyManager}; el módulo Clasificador de Entidades está representado por la clase \emph{ContentClasification}; el módulo que genera el Árbol de Entidades está implementado por la clase \emph{GeneratorTreeManager}, y el módulo que crea nuevos grupos de entidades está implementado en la clase \emph{ContentGrouping}. 

\begin{figure}
    \centering
    \includegraphics{img/organizacion_informacion/clases_organizador_informacion.pdf}
    \caption{Diagrama de clases del organizador de información}
    \label{fig:diagrama_clases_organizador}
\end{figure}

\section{Implementación}
\subsection{blabla}

\section{Casos de estudio}
En esta sección se presentan tres casos de estudio que muestran el comportamiento de la aplicación ante tres entradas diferentes, abarcando distintos aspectos de la solución propuesta.

Al momento de realizar este trabajo no se reconoce ninguna aplicación que aborde el problema de organización de la información en una ontología, por lo que se expondrán tres casos de estudio sin la posibilidad de compararlos con otros resultados.

\subsection{Organización de Ontología Pizza}
Según la descripción de la ontología, este dominio representa las pizzas y sus coberturas, por lo que esperamos que las entidades que representen a las pizzas, a las coberturas y la información complementaria tiendan a agruparse en los niveles más altos de la jerarquía, mientras que en los niveles más bajas esperamos encontrar las clases más específicas y que tienen menor impacto sobre el entendimiento del dominio.

En la Figura~\ref{fig:caso_estudio_pizza} se puede ver el resultado de aplicar el algoritmo teniendo como entrada la ontología pizza. 

Los nombres de las entidades fueron traducidos al español, para poder generar las sentencias en lenguaje español en el siguiente capítulo, pero el idioma no afecta los resultados del algoritmo propuesto.

\begin{figure}
\begin{multicols}{2}
\begin{figure}[H]
\dirtree{%
.1 Pizza.
.2 Pizza con nombre.
.3 Margherita.
.3 Frutti di mare.
.3 Giardiniera.
.3 ....
.3 Napoletana.
.2 Pizza con carne.
.2 Pizza picante.
.2 ....
.2 Pizza vegetariana.
.2 Ingredientes.
.3 Coberturas.
.4 Cobertura de pizza.
.5 Cobertura de verduras.
.6 Cobertura de pimiento.
.6 ....
.6 Cobertura de langostinos.
.3 Bases.
.4 Base de pizza.
.5 Base gruesa.
.5 Base delgada y crujiente.
.5 Condimentoes.
.6 Sales.
.7 Sal.
}
\end{figure}

\begin{figure}[H]
\dirtree{%
.1 Otras secciones.
.2 Domain concept.
.3 Comida.
.4 Helado.
.4 Ingrediente.
.3 Pais.
.4 America.
.4 England.
.4 Italy.
.4 France.
.4 Germany.
.2 Value partition.
.3 Picante.
.4 Poco picante.
.4 Algo picante.
.4 Muy picante.
}
\end{figure}

\end{multicols}
\caption{Resultado del organizador de información con la ontología Pizza.}
\label{fig:caso_estudio_pizza}
\end{figure}

La ontología pizza cuenta con 100 clases y 8 propiedades. De las 8 propiedades, 5 fueron usadas  para clasificar las 100 clases (de las 8, 3 eran inversas a otras 3 por lo que no agregaban información nueva). El promedio de información obtenido fue 1.25, siendo la clase Pizza la única en superar este valor, lo que concuerda con el resultado esperado.

Como se puede observar en la Figura~\ref{fig:caso_estudio_pizza}, en la primer columna se encuentran los tópicos principales reconocidos por el algoritmo, y en la segunda columna se encuentran los tópicos agregados con el módulo para agregar información marginada.

En la jerarquía de los tópicos principales, se puede ver como toda la información se agrupa como hijas de la entidad Pizza, describiendo los tipos de pizzas y sus ingredientes. Inmediato a los ingredientes lista las coberturas, y reconoce a las bases de pizza a la misma altura que las coberturas, asignándoles la misma importancia.

En la jerarquía de \emph{otras secciones}, se aprecian las demás entidades que aportan información secundaria a la descripción del dominio, que no está directamente relacionada con el dominio de las pizzas, como son los países, los tipos de picante y otras comidas.

Respecto al resultado esperado, la organización de la información es satisfactoria, ya que se asimila a la propia descripción del dominio. 

\subsection{Organización de Ontología Wine}
Esta ontología tiene como objetivo describir un dominio de vinos y comidas\footnote{\url{https://protege.stanford.edu/publications/ontology_development/ontology101-noy-mcguinness.html}}, por lo que esperamos que las entidades que se consideren más relevantes sean aquellas afines a los vinos y comidas.

En la Figura~\ref{fig:caso_estudio_wine} se puede ver el resultado de aplicar el algoritmo teniendo como entrada la ontología Wine.

\begin{figure}
\begin{multicols}{2}
{\small
\begin{figure}[H]
\dirtree{%
.1 Wine.
.2 Italian wine.
.3 Chianti.
.4 Chianti classico.
.2 $...$ \emph{(otras subclases de Wine)}.
.2 Wines descriptor.
.3 Sugar.
.4 Wine sugar.
.5 Dry.
.5 Off dry.
.5 Sweet.
.3 Colors.
.4 Wine color.
.5 White.
.5 Rose.
.5 Red.
.3 Flavors.
.4 Wine flavor.
.5 Moderate.
.5 Strong.
.5 Delicate.
.3 Bodies.
.4 Wine body.
.5 Medium.
.5 Full.
.5 Light.
.2 From grapes.
.3 Wine grape.
.4 Chenin blanc grape.
.4 $...$ \emph{(otras subclases de Wine grape)}.
}
\end{figure}}  

\begin{figure}[H]
\dirtree{%
.1 Otras secciones.
.1 Region.
.2 Central texas region.
.2 $...$ \emph{(otras subclases de Region)}.
.1 Vintage year.
.2 Year1998.
.1 Wine descriptor.
.2 Wine taste.
.1 Non consumable thing.
.1 Vintage.
.2 Vintage years.
.1 Fruit.
.1 Winery.
.2 Chateau de meursault.
.2 $...$ \emph{(otras subclases de Winery)}.
.1 Consumable thing.
.2 Meal.
.3 Courses.
.4 Meal course.
.5 Cheese nuts dessert course.
.5 $...$ \emph{(otras subclases de Meal course)}.
.5 Foods.
.6 Edible thing.
.7 Fowl.
.8 $...$.
.7 Dessert.
.8 $...$.
.7 Meat.
.8 $...$.
.7 Seafood.
.8 $...$.
.7 Sweet fruit.
.8 Grape.
.8 $...$.
.7 Pasta.
.8 $...$.
.7 Non sweet fruit.
.5 Drinks.
.6 Potable liquid.
.7 Juice.
.7 Wine.
.8 Wines descriptor.
.8 From fruits.
}
\end{figure}

\end{multicols}
\caption{Resultado del organizador de información con la ontología Wine.}
\label{fig:caso_estudio_wine}
\end{figure}


La ontología Wine cuenta con 138 clases y 16 propiedades. De las 16 propiedades, 12 fueron usadas para verificar qué clases superan el umbral promedio para ser seleccionadas como las principales. El promedio fue de 2.5 y la única clase que lo superó fue Wine, lo que cumple parcialmente el resultado esperado, ya que parte del objetivo de la ontología es describir el dominio de los \emph{Wines}. Respecto a la sección que describe las comidas, quedó desplazada a \emph{otras secciones}, siendo un resultado no esperado según el objetivo de la ontología. Sin embargo, analizando manualmente la ontología, se puede apreciar que el porcentaje de  información que describe a las comidas es significativamente menor en relación a la información referida a los vinos, factor por el cual resulta aceptable que no aparezcan como sección principal.

%\subsection{Organización de Ontología SNOMED-CT}

\section{Conclusiones}

\chapter{Generación del documento de texto}

\section{Diseño}
En este capítulo se diseñará e implementará el proceso de generación del documento de texto. Elegimos organizar toda la información de la ontología generando un documento dividido en secciones, subsecciones y párrafos. Para alcanzar el documento final, se crearon módulos teniendo como referencia las actividades de la generación de lenguaje natural nombradas en la Sección \ref{sec:tareas_gnl}. Los tres módulos principales son \emph{Macro planificación}, \emph{Micro planificación} y \emph{Realización}. Dentro de cada módulo se desarrollaron submódulos, para resolver cada problema específico. 

Para comenzar con la generación de un documento de texto, se planteó un Plan de Documento. Este Plan tiene la estructura resultante del Organizador de la Información. Sin embargo, para crear el documento en la etapa de Macro Planificación, se reemplazó el módulo Generador del Árbol de Entidades del Organizador de Información, por un nuevo módulo específico del Macro Planning. El motivo de este reemplazo es evitar la creación de interfaces entre el Organizador de Información y el Macro Planning, por el overhead que estas suponen, ya que la creación del Árbol de Entidades y la Macro Planificación resultan ser procesos equivalentes, por lo que pueden ser reemplazables, evitando recorrer el doble de veces la estructura jerárquica.

En la figura~\ref{fig:modulos_plan_documento} se puede ver el nuevo módulo Generador del Plan de Documento, que retorna como salida un Plan de Documento.

\begin{figure}[H]
    \centering
    \includegraphics{img/generacion_documento/modulos_plan_documento.pdf}
    \caption{Módulos para la generación del Plan de Documento.}
    \label{fig:modulos_plan_documento}
\end{figure}

Cabe aclarar que este es un Plan de Documento inicial que solo cuenta con una primera distribución de la información que estará contenida en el texto final. En el Documento Final, la realización de la estructura y la realización lingüística (secciones, párrafos y oraciones) se realizarán en las etapas de Micro Planificación y Realización, como se ve en la figura~\ref{fig:modulos_documento_final}.

\begin{figure}[H]
    \centering
    \includegraphics[width=12cm]{img/generacion_documento/modulos_documento_final.pdf}
    \caption{Módulos para la generación del documento final.}
    \label{fig:modulos_documento_final}
\end{figure}

\section{Diseño Macro Planning}
\label{sec:macro_planning}
En esta etapa se planificará la estructura del documento. Como nombramos anteriormente, la organización a priori de la información, realizada en el capítulo anterior, facilita la tarea de Macro Planificación. Por este motivo, se decidió hacer una arquitectura monolítica entre el Organizador de Información y el Macroplanificador. 

El documento de texto estará representado internamente con una estructura basada en componentes. Esta representación prepara el terreno para que, luego de la Micro Planificación, el Realizador decida el diseño final del documento.

La representación interna del documento de texto tiene en cuenta los siguientes criterios:
\begin{itemize}
    \item Cada Entidad presente en la estructura que resulta del Organizador de Información es considerada un tópico. 
    \item A cada tópico del primer nivel le corresponde una sección. Los tópicos anidados (Subclases, Individuos y Relaciones) se corresponden a subsecciones (secciones anidadas).
    \item Cada sección está compuesta de al menos un párrafo y un título.
    \item Cada párrafo trata un único tópico, el cual puede ser una clase o un individuo. También, un párrafo puede o no tener alguna sentencia. Esto se debe a que no todas las entidades tienen asociados axiomas que la describan.
\end{itemize}

\subsection{Secciones y párrafos}
Las Secciones tiene como objetivo principal delimitar la información asociada a una Entidad, segmentando el texto con el fin de mantener la coherencia global.

Dentro de una sección para una Clase, se verbalizarán sus axiomas y se enumerarán sus Individuos, Subclases y Relaciones, todo en un párrafo. Enumerar esta información resulta conveniente ya que luego se crearán subsecciones para cada uno de esos componentes.

En una sección asociada a una Relación, solo habrán subsecciones acerca de las Entidades que pertenecen a la Relación.

En una sección para Individuos, se creará un párrafo para describir sus propiedades.

\subsubsection{Subsecciones}
Las subsecciones son secciones anidadas dentro de otras secciones. Una subsección puede surgir a partir de un Individuo, Subclases o Relaciones. Cuando surge a partir de un Individuo, la subsección trata como tópico al Individuo; cuando surge a partir de una Subclase, el tópico principal es la Subclase; y cuando surge a partir de una Relación, el tópico principal resulta ser el rango de la Relación, es decir, otra Clase.

\subsection{Oraciones}
Las oraciones pertenecen a los párrafos por lo que tratan un solo tópico. Cada oración que tenga como tópico a una Clase,  aborda un solo tipo de información, por lo que existe una oración para las clases disjuntas, una oración para las clases equivalentes, y así con los demás tipos de información. Cada oración correspondientes a un Individuo contiene las propiedades con sus valores declaradas sobre el Individuo.

Al tratar un solo tipo de información por oración, se maximiza la cohesión dentro de cada oración.


\subsection{Diagrama de clases}
En la figura \ref{fig:diagrama_clases_macroplanificador} se muestra el diagrama de clases correspondiente al Macroplanificador. La clase TextManager es la implementación del módulo Generador del Plan de Documento. Esta clase se encarga de crear la clase Text, que es la representación interna de un documento de texto. El documento está compuesto por Secciones, Párrafos y Oraciones, cada uno de los cuales tiene asociado su respectiva clase Section, Paragraph y Statement. Las clases StatementComponent y Word son específicas del Micro Planning por lo que se explicarán más adelante.

\begin{figure}
    \centering
    \includegraphics{img/generacion_documento/diagrama_clases_macroplanificador.pdf}
    \caption{Diagrama de clases del Macro Planning.}
    \label{fig:diagrama_clases_macroplanificador}
\end{figure}

\section{Implementación Macro Planning}
La clase principal del Macroplanificador es \emph{TextManager}. Para crear el Documento Inicial, recorre las Entidades de la misma manera que la clase \emph{GeneratorTreeManager}, explicada en la Figura~\ref{fig:clase_generator_tree}, con la diferencia que crea el objeto \emph{Text} y los objetos \emph{Section} asociados a cada Entidad. Ya que el recorrido es equivalente no se presentará su implementación, en cambio comenzaremos a mostrar las clases que componen el Documento de Texto.

La clase \emph{Text} solo contiene los objetos del primer nivel del Documento de Texto. Tiene acceso a las secciones principales y a las secciones que contienen la información marginada. 

La clase \emph{Section} contiene la información referida a una Sección, como título, párrafos, sus subsecciones y el tópico asociado. Implementa el método de realización del Documento Final, el cual se verá en el Capítulo de Realización.

Cada \emph{Section} se encarga de crear el \emph{Paragraph} correspondiente a su tópico.

La clase \emph{Paragraph} contiene una lista de todas las \emph{Statement} referidas a su Tópico. 

Cada \emph{Paragraph} se encarga de crear las \emph{Statement} correspondientes para el Tópico que recibe, y asociarles los componentes que describen al Tópico. Adicionalmente el párrafo recibe el lenguaje del texto, por lo que debe instanciar las \emph{Statement} adecuadas, ya sea \emph{StatementEnglish} o \emph{StatementSpanish}.

La clase \emph{Statement} contiene la información que tendrá cada oración. Será usada por el Microplanificador para verbalizar la información e implementar los métodos de la etapa de Micro Planning. 

\section{Diseño Micro Planning}
En esta estapa se llevarán a cabo las tareas crear las oraciones del documento de texto.
para describir una Entidad en lenguaje natural a partir de la información que contiene cada oración creada en el Macro Planning. Esta descripción en lenguaje natural se alcanza a través de una o varias oraciones producidas con una estructura que se adapte a la gramática del lenguaje español.

Como existen diversas estructuras sintácticas para expresar un mismo contenido, a modo de simplificación se asoció a cada constructor OWL solo algunas formas de verbalización. Se trató de generar oraciones que sean fieles al significado expresado en los axiomas y de evitar ambigüedades.

Para poder trabajar con patrones de la gramática del lenguaje humano (ya sea español o inglés), se utilizó un \emph{POS Tagger} para etiquetar las palabras.

\subsection{Verbalización de una oración}
Tanto las oraciones acerca de Clases como las de Individuos requieren la verbalización de los constructores OWL. Solo en el caso de las oraciones acerca de Clases requieren además la enumeración de sus instancias, propiedades y subclases. Con enumeración de las propiedades (y de instancias y subclases por igual), nos referimos únicamente a presentarlas, por ejemplo, las propiedades \emph{tieneCobertura} y \emph{tieneBase} para la clase Pizza resultan en la oración ``una pizza tiene base de pizza y tiene cobertura de pizza''.

Las tareas necesarias para la verbalización de una oración son:
\begin{itemize}
    \item Verbalizar los constructores OWL o enumerar las propiedades, instancias y subclases.
    \item Eliminar información redundante entre oraciones a través del proceso de agragación.
    \item En el caso de las Clases, se reemplaza con expresiones de referencia a los nombres de las Clases en oraciones consecutivas evitando que se repitan y que genere un texto poco fluido.
\end{itemize}

\subsection{Sintaxis OWL 2}
\label{sec:gen_doc_sintaxis_owl}
Teniendo en cuenta la estructura jerárquica de la sintaxis (a partir de su gramática BNF y de los diagramas de clases de su documentación~\cite{OWL2W3C}), agrupamos en diferentes niveles de abstracción las tres categorías sintácticas: en el nivel inferior se encuentran las Entidades, en el nivel intermedio las Expresiones de Clases, y en el nivel superior los Axiomas. Esta separación en niveles de abstracción será útil para organizar la verbalización. El conjunto de constructores elegido para verbalizar se encuentra en las siguientes listas: en la lista~\ref{list:constructores_axiomas} se agrupan los constructores del nivel de Axiomas, en la lista~\ref{list:constructores_expresiones} los constructores de Expresiones de Clases y en la lista~\ref{list:constructores_entity} los constructores de nivel de Entidades.
\begin{figure}
\begin{multicols}{2}
\captionof{listCap}{Constructores de Entidades}
\label{list:constructores_entity}
    \begin{itemize}
        \item owl:class
        \item owl:objectProperty
        \item owl:dataProperty
        \item owl:individual
        \item[\vspace{\fill}]
    \end{itemize}

\captionof{listCap}{Constructores de Axiomas}
\label{list:constructores_axiomas}
    \begin{itemize}
        \item rdfs:subClassOf
        \item owl:equivalentClass
        \item owl:disjointWith
        \item rdfs:domain
        \item rdfs:range
    \end{itemize}
    \end{multicols}
\end{figure}


\begin{figure}
\captionof{listCap}{Constructores de Expresiones de Clase}
\label{list:constructores_expresiones}
    \begin{itemize}
    \begin{multicols}{2}
        \item owl:intersectionOf
        \item owl:unionOf
        \item owl:complementOf
        \item owl:allValuesFrom
        \item owl:someValuesFrom 
        \item owl:minCardinality
        \item owl:maxCardinality
        \item owl:Cardinality
        \item owl:oneOf
        \item owl:hasValue
        \end{multicols}
    \end{itemize}
\end{figure}

\subsection{De OWL 2 a lenguaje natural}
La verbalización de la sintaxis OWL 2 se corresponde a la tarea de traducir el lenguaje OWL al lenguaje humano. Para llevar a cabo la traducción, se tendrán en cuenta los niveles de abstracción presente en la Sección~\ref{sec:gen_doc_sintaxis_owl}. 

El enfoque adoptado para crear la sintaxis de las oraciones se basa en un recorrido bottom-up de la jerarquía de un Axioma. Se comienza desde las hojas, construyendo oraciones parciales a partir de las Entidades, luego se procede a subir por los niveles a través de los constructores de Expresiones de Clases, componiendo nuevas oraciones parciales, hasta alcanzar la raíz de la jerarquía, donde se termina de construir la oración final. Adicionalmente, se realizan algunos tratamientos morfológicos para agregar fluidez y coherencia al texto, tal como insertar artículos, y hacer concordar los sustantivos en género y número.

Con el objetivo de comenzar a formar las oraciones parciales, el primer paso a realizar es verbalizar las Entidades, ya que están directamente relacionadas con las IRIs (o tienen acceso al \emph{label}, en caso de extraer los nombres desde los \emph{labels}).

En la Gramática~\ref{gram:entity} se muestra una porción de la gramática BNF del lenguaje OWL 2 asociada a las Entidades.
\begin{GrammarEnv}
%detecta un error de sintaxis pero es mentira.
\begin{grammar}
[(colon){$\rightarrow$}]
[(semicolon)$|$]
[(comma){}]
[(period){\vspace{0.3cm} \\}]
[(quote){\begin{bf}}{\end{bf}}]
[(nonterminal){$<$}{$>$}]
%<expression> : <number> ; <number>, [\{"asd"\}], <relational\_operator>, <number>.
%<number> : <digit> ; <digit> , <number>.
%<digit> : "0";"1";"2";"3";"4";"5";"6";"7";"8";"9".
%<relational\_operator> : $"="$;"$\lessthan \greaterthan$";"$\lessthan$";"$\greaterthan$"; "$\lessthan=$";"$\greaterthan=$";"in".
\fbox{\begin{minipage}{14cm}
<Entity> : <Class> ; <ObjectProperty> ; <DataProperty> ; <NamedIndividual> ; <AnnotationProperty>.
<Class> : <IRI> .
<ObjectProperty> : <IRI> .
<DataProperty> : <IRI> .
<AnnotationProperty> : <IRI> .
<NamedIndividual> : <IRI> .
\end{minipage}
}
\end{grammar}
\caption{Porción de gramática asociada a las Entidades.}\label{gram:entity}
\end{GrammarEnv}

En el siguiente nivel se tienen los constructores de Expresiones de Clase, con las cuales es posible generar oraciones complejas subordinando las expresiones que se encuentran anidadas, o componer nuevas oraciones parciales teniendo en cuenta los tipos de componentes involucrados.

En la Gramática~\ref{gram:expr_clases} se muestra a modo de ejemplo una sección de la gramática con constructores de este nivel.
\begin{GrammarEnv}
\begin{grammar}
[(colon){$\rightarrow$}]
[(semicolon)$|$]
[(comma){}]
[(period){\vspace{0.3cm} \\}]
[(quote){\begin{bf}}{\end{bf}}]
[(nonterminal){$<$}{$>$}]
%<expression> : <number> ; <number>, [\{"asd"\}], <relational\_operator>, <number>.
%<number> : <digit> ; <digit> , <number>.
%<digit> : "0";"1";"2";"3";"4";"5";"6";"7";"8";"9".
%<relational\_operator> : $"="$;"$\lessthan \greaterthan$";"$\lessthan$";"$\greaterthan$"; "$\lessthan=$";"$\greaterthan=$";"in".
\fbox{\begin{minipage}{14cm}
<ClassExpression> : <Class> ; <ObjectIntersectionOf> ; <ObjectSomeValuesFrom> .
<ObjectSomeValuesFrom> : "ObjectSomeValuesFrom" "(" <ObjectPropertyExpression> <ClassExpression> ")" .
<ObjectPropertyExpression> : <ObjectProperty> .
\end{minipage}}
\end{grammar}
\caption{Porción de gramática asociada a las Expresiones de Clases.}\label{gram:expr_clases}
\end{GrammarEnv}

Puede apreciarse en la gramática que existe recursividad entre los constructores, lo que produce que un constructor aparezca en diferentes niveles y pueda combinarse con cualquier otro constructor del nivel de Expresiones de Clases. En este punto es importante que cada constructor pueda componerse con los demás, para asegurar cualquier combinación posible.

Por último en el nivel más abstracto se encuentran los constructores de Axiomas. Estos constructores tienen la característica de no ser recursivos entre ellos, por lo que es posible generar oraciones independientes, yuxtapuestas o coordinadas. Un ejemplo de estos constructores se ve en la Gramática~\ref{gram:axiom}.

\begin{GrammarEnv}
\begin{grammar}
[(colon){$\rightarrow$}]
[(semicolon)$|$]
[(comma){}]
[(period){\vspace{0.3cm} \\}]
[(quote){\begin{bf}}{\end{bf}}]
[(nonterminal){$<$}{$>$}]
%<expression> : <number> ; <number>, [\{"asd"\}], <relational\_operator>, <number>.
%<number> : <digit> ; <digit> , <number>.
%<digit> : "0";"1";"2";"3";"4";"5";"6";"7";"8";"9".
%<relational\_operator> : $"="$;"$\lessthan \greaterthan$";"$\lessthan$";"$\greaterthan$"; "$\lessthan=$";"$\greaterthan=$";"in".
\fbox{\begin{minipage}{14cm}
<ClassAxiom> : <SubClassOf> ; <EquivalentClasses> ; <DisjointClasses> ; <DisjointUnion> .
<EquivalentClasses> : "EquivalentClasses" "(" <axiomAnnotations> <ClassExpression> <ClassExpression> \{ <ClassExpression> \} ")".
\end{minipage}}
\end{grammar}
\caption{Porción de gramática asociada a los Axiomas.}\label{gram:axiom}
\end{GrammarEnv}

\subsection{Componentes y oraciones parciales}
Durante la creación de una oración, se recorren los constructores de Expresiones de Clases y de Entidades, creando oraciones parciales y componiéndolas entre ellas. Dado que cada constructor puede componerse con cualquier otro y en cualquier nivel de profundidad, se decidió asociar a cada constructor un tipo de componente oracional, de esta manera se evita una exhaustiva programación de composiciones.
Los tipos de componentes son los siguientes:
\begin{itemize}
    \item Término (T): caracterizado por no poseer verbo. Pueden contener adverbios, sustantivos y adjetivos.
    \item Sintagma Verbal (SV): debe poseer un verbo.
    \item Oración Negativa (ON): representa la negación de un componente.
    \item Unión: representa una disyunción de oraciones parciales.
    \item Intersección: representa una adición o subordinación de oraciones parciales.
\end{itemize}

Estos tipos de componentes definen un sistema de tipos, en el que cada tipo puede componerse con otro y dan como resultado un nuevo componente. La particularidad de este sistema es que el tipo de una composición no depende de sus componentes, sino del constructor que opere con esos componentes.
%En la mayoría de los casos el tipo de componente es predecible, a excepción de la \emph{intersección}.

Los componentes que retorna cada constructor se definen a continuación:
\begin{itemize}
    \item Componente Término (T):
    \begin{itemize}
        \item Clase.
        \item Individuo
    \end{itemize}
    \item Componente Sintagma Verbal (SV):
    \begin{itemize}
        \item Propiedad.
        \item Constructores de cuantificación.
        \item Constructor \emph{hasValue}.
        \item Constructores con cardinalidad.
    \end{itemize}
    \item Componente Union: 
    \begin{itemize}
        \item \emph{UnionOf}.
        \item \emph{OneOf}
    \end{itemize}
    
    \item Componente Intersección: solo el constructor \emph{IntersectionOf}.
    \item Componente Oración Negativa (ON): solo el constructor \emph{ComplementOf}.
\end{itemize}

El beneficio de este sistema es que cada constructor sabe cómo realizar la composición, en función del tipo de cada uno de sus operandos, lo que reduce la cantidad de combinaciones a programar (en contraste con tener que programar el proceso de composición de cada constructor teniendo en cuenta que sus operandos serían otros constructores en lugar de componentes). 

Además, incorporar un sistema como este, evita el uso de otras técnicas menos afines a la lingüística, como \emph{templates} prediseñados o texto genérico para cada caso particular, ya que se basa en el uso de patrones más cercanos a la Lingüística Teórica, apoyando el uso de teorías lingüísticas en el campo del Procesamiento de Lenguaje Natural.


Sin embargo, estos componentes no intentan ser exhaustivos ni completamente fieles a la sintaxis del lenguaje humano, sino que intentan acaparar de la forma más general los posibles resultados de cada constructor, para mantener una sintaxis sencilla y aceptable. Una Clase, por ejemplo, podría cumplir la función de sintagma nominal, adjetival o adverbial, pero para simplificar la nomenclatura, se retorna un tipo más genérico y al momento de usar una Clase, de ser necesario, se resuelve la composición en función de los elementos que la constituyen. Como veremos más adelante, a veces es innecesario discriminar los tipos de sintagmas.


\subsection{Convención y suposiciones del nombrado de Entidades}
Para no limitar la aplicación a una convención de nombrado, no se requiere una forma estricta para nombrar las entidades en una ontología. Sin embargo, nos basamos en algunas suposiciones que resultan natural al crear una ontología:
\begin{enumerate}
    \item Las entidades pueden tener su nombre ya sea en la IRI o en un label\footnote{Estas opciones son excluyentes, al verbalizar una ontología solo se extraen los nombres de un solo lugar, por lo que todos los nombres deben ser extraídos de IRI o de label sin combinarse}. Si el nombre es extraído de una IRI, debe estar escrito con el estilo CamelCase. Si el nombre se extrae de un label, debe estar escrito en lenguaje natural, y debe tener el idioma del label (en este caso, español ``es'' o ingles ``en'').
    \item No hay condiciones para el nombre de una clase, sin embargo es preferible que contenga al menos un sustantivo.
    \item No hay condiciones para el nombre de una propiedad, sin embargo es preferible que contenga al menos un verbo. 
    
    No se infieren verbos, por lo que si se quiere expresar, por ejemplo, que una clase \emph{está localizada en} algún lugar, la propiedad debería llamarse \emph{estáLocalizadaEn} y no \emph{localizadaEn}. Al igual que los verbos, no es conveniente omitir las preposiciones finales, por ejemplo, en la propiedad \emph{esParte}, es recomendable utilizar \emph{esParteDe}.
\end{enumerate}

Estas condiciones ayudan a mejorar la fluidez del texto y no suponen una gran carga en los desarrolladores de ontologías.


\subsection{Tratamiento sintáctico de los nombres de Entidades}
Para facilitar el entendimiento de las secciones referidas al micro planning, se explicará cómo fueron tratados los nombres de las entidades, desde el punto de vista sintáctico. 

Dado que los nombres de las Entidades pueden no corresponderse fielmente con la gramática de un lenguaje (por falta de palabras funcionales, tales como artículos o preposiciones), se optó por no realizar un análisis sintáctico a los nombres. Sin embargo, para soportar la creación de patrones gramaticales, se etiquetaron las palabras de cada Entidad para reconocer la función gramatical de cada una y poder llevar a cabo las composiciones.

Se consideró, a cada conjunto de palabras que conforman el nombre de una entidad, como un componente incompleto (oración parcial) que es parte de una oración más grande. Por este motivo, se divide en dos partes fundamentales: una parte inicial y un complemento, entre los cuales pueden insertarse nexos y cuantificaciones. A continuación se explican cómo se reconocen ambas partes en las clases y propiedades:
\begin{itemize}
    \item Los nombres de las propiedades (en los que suponemos existe al menos un verbo), se considera como parte inicial a todas las palabras desde el inicio hasta el primer verbo, incluído el verbo. El complemento se conforma con todas las palabras que siguen al verbo. Por ejemplo: la propiedad \emph{tiene color} se compone de parte inicial: \emph{tiene}, y complemento: \emph{color}. La propiedad \emph{se solapa con} se compone de \emph{se solapa} como parte de inicial, y \emph{con} como complemento.
    \item Los nombres de las clases (en los que suponemos existe al menos un sustantivo), se considera como parte inicial a todas las palabras desde el inicio hasta el primer sustantivo, incluido el sustantivo. El complemento se conforma con todas las palabras que siguen al sustantivo. Por ejemplo: la clase \emph{agente social}, tiene como parte inicial \emph{agente} y complemento \emph{social}. La clase \emph{cobertura de pizza} se compone de \emph{cobertura} con complemento \emph{de pizza}.
    \item Los nombres de los individuos son tratados igual que los nombres de las clases.
\end{itemize}


\subsection{Verbalización de constructores OWL}
\label{sec:verbalizacion_constructores}
Esta tarea se encarga de componer las oraciones según la información de cada constructor OWL.
%Las oraciones pueden enfocarse ya sea en una \emph{owl:Class} o  \emph{owl:NamedIndividual}. Cuando se enfocan en una \emph{owl:Class}, pueden contener la siguiente información: clases equivalentes, disjuntas y superclases. Cuando se enfoca en un \emph{owl:NamedIndividual}, puede contener información acerca de los valores de sus propiedades.
%Para cada uno de estos tipos de información, se pueden presentar una o varias oraciones. Las oraciones que tratan el mismo tipo de información se organizan adyacentemente entre ellas.
En algunos constructores, tener un único patrón gramatical resulta insuficiente, debido a que pueden recibir distintas estructuras gramaticales como entrada, que requieren diferentes formas de componerse. Por lo tanto, en estos casos se diseñó más de una forma de composición, mejorando la variabilidad, fluidez e interpretación de las oraciones. 

Para reconocer cómo componer las oraciones en cada constructor, se revisó empíricamente los axiomas de algunas ontologías, y se buscó utilizar oraciones que sean lo más genéricas posibles, que permitan la comprensión de los axiomas. 

A continuación se explican los patrones gramaticales usados para cada constructor. %El signo $+$ corresponde a la concatenación de cadenas de texto.

\subsubsection{Restricción de cardinalidad \emph{ObjectCardinalityRestriction}}

La gramática~\ref{gram:object_card_rest} muestra los patrones diseñados para las restricciones de cardinalidad sobre \emph{objectProperty}. En los casos donde sea posible, si la clase sobre la que se aplica la restricción es \emph{owl:Thing}, se reemplaza por el rango de la propiedad, el complemento de la propiedad (como el patron$_2$) o por último por la palabra ``cosa''.

El enlace es seleccionado dependiendo del operador \emph{MinCardinality}, \emph{MaxCardinality} y \emph{ExactCardinality}.

Ejemplo: para la clase ``pizza interesante'', se tiene el axioma: ``tieneCobertura min 3 owl:Thing'', que puede verbalizarse como ``tiene como mínimo 3 coberturas de pizza''. La parte de la oración que se corresponde a \emph{coberturas de pizza}, es extraída del rango de la propiedad.

\begin{GrammarEnv}
\begin{grammar}
[(colon){$\rightarrow$}]
[(semicolon)$|$]
[(comma){}]
[(period){\vspace{0.3cm} \\}]
[(quote){\begin{bf}}{\end{bf}}]
[(nonterminal){$<$}{$>$}]
\fbox{\begin{minipage}{14cm}
<RestriccionCard> : <patron$_1$> ; <patron$_2$> ; <patron$_3$>.
<patron$_1$> : <parteInicialPropiedad> <complementoPropiedad> <enlace> <cardinalidad> <clases> .
<patron$_2$> : <parteInicialPropiedad> <enlace> <cardinalidad> <complementoPropiedad>.
<patron$_3$> : <parteInicialPropiedad> <enlace> <cardinalidad> <rangoPropiedad>.
<patron$_4$> : <parteInicialPropiedad> <enlace> <cardinalidad> <clases>.
<enlace> : "como mínimo" ; "como máximo"; "exactamente".
\end{minipage}}
\end{grammar}
\caption{Patrones para ObjectCardinalityRestriction.}\label{gram:object_card_rest}
\end{GrammarEnv}

\subsubsection{Restricción de cardinalidad \emph{DataCardinalityRestriction}}
La gramática~\ref{gram:data_card_rest} muestra los patrones diseñados para las restricciones de cardinalidad sobre \emph{dataProperty}. El patron$_2$ es específico para cuando la propiedad no tiene verbo. 

\begin{GrammarEnv}
\begin{grammar}
[(colon){$\rightarrow$}]
[(semicolon)$|$]
[(comma){}]
[(period){\vspace{0.3cm} \\}]
[(quote){\begin{bf}}{\end{bf}}]
[(nonterminal){$<$}{$>$}]
\fbox{\begin{minipage}{14cm}
<RestriccionCard> : <patron$_1$> ; <patron$_2$>.
<patron$_1$> : <parteInicialPropiedad> <enlace> <cardinalidad> <complementoPropiedad> .
<patron$_2$> : "tiene" <enlace> <cardinalidad> <nombrePropiedad>.
<enlace> : "al menos" ; "como máximo"; "exactamente".
\end{minipage}}
\end{grammar}
\caption{Patrones para DataCardinalityRestriction.}\label{gram:data_card_rest}
\end{GrammarEnv}

\subsubsection{Restricción de cuantificación \emph{QuantifiedObjectRestriction}}
La gramática~\ref{gram:quant_obj_rest} muestra los patrones diseñados para las cuantificaciones sobre \emph{objectProperty}.

\begin{GrammarEnv}
\begin{grammar}
[(colon){$\rightarrow$}]
[(semicolon)$|$]
[(comma){}]
[(period){\vspace{0.3cm} \\}]
[(quote){\begin{bf}}{\end{bf}}]
[(nonterminal){$<$}{$>$}]
\fbox{\begin{minipage}{14cm}
<RestriccionQuant> : <patron$_1$> ; <patron$_2$> ; <patron$_3$> ; <patron$_4$> ; <patron$_5$> .
<patron$_1$> : <verboPropiedad> "como" <complementoPropiedad> <enlace> <clases> .
<patron$_2$> : <verboPropiedad> <enlace> <clases> .
<patron$_3$> : <verboPropiedad> <enlace> "a" <clases> .
<patron$_4$> : <verboPropiedad> "a" <enlace> <clases> .
<patron$_5$> : <verboPropiedad> <complementoPropiedad> <enlace> <clases> .
<enlace> : "algún" ; "alguna"; "algo"; "exclusivamente" .
\end{minipage}}
\end{grammar}
\caption{Patrones para QuantifiedObjectRestriction.}\label{gram:quant_obj_rest}
\end{GrammarEnv}

Los enlaces dependen del tipo de cuantificador y del tipo de palabra que lo proceda. Si el cuantificador es existencial, el enlace sería ``alguna/algún'' para palabras que sean sustantivo femenino o masculino respectivamente, o con cualquier otra palabra sería ``algo''.  Para el cuantificador universal el enlace es siempre ``exclusivamente''.

Ejemplo: la clase ``Margherita'' que es subclase de ``PizzaConNombre'', tiene el axioma ``tieneCobertura only 
    (CoberturaDeMozzarella or CoberturaDeTomate)'' (siendo only el cuantificador universal), el cual se traduce a ``tiene cobertura de mozzarella o tomate''. También posee dos axiomas equivalentes al anterior: ``tieneCobertura some CoberturaDeMozzarella'' y ``tieneCobertura some CoberturaDeTomate'', (siendo some el cuantificador existencial), los cuales se traducen a ``tiene alguna cobertura de mozzarella'' y ``tiene alguna cobertura de tomate''.


\subsubsection{Restricción de cuantificación \emph{QuantifiedDataRestriction}}
La gramática~\ref{gram:quant_data_rest} muestra los patrones diseñados para las cuantificaciones sobre \emph{dataProperty}.

\begin{GrammarEnv}
\begin{grammar}
[(colon){$\rightarrow$}]
[(semicolon)$|$]
[(comma){}]
[(period){\vspace{0.3cm} \\}]
[(quote){\begin{bf}}{\end{bf}}]
[(nonterminal){$<$}{$>$}]
\fbox{\begin{minipage}{14cm}
<RestriccionQuant> : <patron$_1$> ; <patron$_2$> .
<patron$_1$> : <verboPropiedad> <enlace> <complementoPropiedad> "de tipo" <rangoPropiedad> .
<patron$_2$> : <verboPropiedad> <complementoPropiedad> <enlace> "de tipo" <rangoPropiedad> .
<enlace> : "algún" ; "alguna"; "algo"; "exclusivamente" .
\end{minipage}}
\end{grammar}
\caption{Patrones para QuantifiedDataRestriction.}\label{gram:quant_data_rest}
\end{GrammarEnv}

\subsubsection{Restricción \emph{OWLObjectComplementOf}}
La gramática~\ref{gram:complement_rest} muestra los patrones diseñados para el complemento de una Expresión de Clase.

\begin{GrammarEnv}
\begin{grammar}
[(colon){$\rightarrow$}]
[(semicolon)$|$]
[(comma){}]
[(period){\vspace{0.3cm} \\}]
[(quote){\begin{bf}}{\end{bf}}]
[(nonterminal){$<$}{$>$}]
\fbox{\begin{minipage}{14cm}
<RestriccionComplement> : <patron$_1$>.
<patron$_1$> : <enlace> <clase> .
<enlace> : "lo opuesto de" ; "no"; "ni"; "excepto" .
\end{minipage}}
\end{grammar}
\caption{Patrones para OWLObjectComplementOf.}\label{gram:complement_rest}
\end{GrammarEnv}

\subsubsection{Restricción \emph{OWLObjectInverseOf}}
La gramática~\ref{gram:inverse_rest} muestra los patrones diseñados para el inverso de una Propiedad.

\begin{GrammarEnv}
\begin{grammar}
[(colon){$\rightarrow$}]
[(semicolon)$|$]
[(comma){}]
[(period){\vspace{0.3cm} \\}]
[(quote){\begin{bf}}{\end{bf}}]
[(nonterminal){$<$}{$>$}]
\fbox{\begin{minipage}{14cm}
<RestriccionInverse> : <patron$_1$>.
<patron$_1$> : <enlace> <Propiedad> .
<enlace> : "es lo opuesto de" .
\end{minipage}}
\end{grammar}
\caption{Patrones para OWLObjectInverseOf.}\label{gram:inverse_rest}
\end{GrammarEnv}

\subsubsection{Restricción \emph{OWLObjectHasValue}}
La gramática~\ref{gram:hasvalue_rest} muestra los patrones diseñados para los valores de una Propiedad.

\begin{GrammarEnv}
\begin{grammar}
[(colon){$\rightarrow$}]
[(semicolon)$|$]
[(comma){}]
[(period){\vspace{0.3cm} \\}]
[(quote){\begin{bf}}{\end{bf}}]
[(nonterminal){$<$}{$>$}]
\fbox{\begin{minipage}{14cm}
<RestriccionHasValue> : <patron$_1$> ; <patron$_2$>.
<patron$_1$> : <parteInicialPropiedad> <complementoPropiedad> <Individuo>.
<patron$_2$> : <parteInicialPropiedad> <Individuo> .
\end{minipage}}
\end{grammar}
\caption{Patrones para OWLHasValue.}\label{gram:hasvalue_rest}
\end{GrammarEnv}

\subsubsection{Restricción \emph{OWLObjectHasSelf}}
La gramática~\ref{gram:hasself_rest} muestra los patrones diseñados para una auto restricción de una Propiedad sobre sí misma.

\begin{GrammarEnv}
\begin{grammar}
[(colon){$\rightarrow$}]
[(semicolon)$|$]
[(comma){}]
[(period){\vspace{0.3cm} \\}]
[(quote){\begin{bf}}{\end{bf}}]
[(nonterminal){$<$}{$>$}]
\fbox{\begin{minipage}{14cm}
<RestriccionHasSelf> : <patron$_1$>.
<patron$_1$> : <parteInicialPropiedad> <complementoPropiedad> "Sí MISMO".
\end{minipage}}
\end{grammar}
\caption{Patrones para OWLObjectHasSelf.}\label{gram:hasself_rest}
\end{GrammarEnv}


\subsubsection{Restricción N-Ary \emph{UnionOf} y \emph{oneOf}} 
Como representan una secuencia de elementos, en la que uno o más elementos pueden ser verdaderos, se eligió concatenar a todos los elementos con el conector disyuntivo ``o''. Sean e1, e2,...,eN los elementos de la enumeración, el patrón gramatical elegido es: ``e1, e2, ..., eN-1 \emph{o} eN''.

\subsubsection{Restricción N-Ary \emph{InterseccionOf}}
La intersección da lugar tanto a la adición como a la subordinación de oraciones. Antes de verbalizar una intersección, se ordenan sus componentes de mayor a menor prioridad, teniendo en cuenta que: T, ON, Unión, Intersección tienen la misma prioridad, y SV tiene menor prioridad. 

Para determinar cómo verbalizar una intersección, se tienen en cuenta las siguientes condiciones:
\begin{itemize}
    \item Si la intersección tiene dos operandos:
        \begin{itemize}
            \item Si el primero es de tipo T  y el segundo SV, se subordina la segunda oración de la siguiente manera:
            ``contenido primer operando + ``que'' + contenido segundo operando''
            \item en cualquier otro caso, se coordinan los operandos:
            ``contenido primer operando + ``y'' + contenido segundo operando''
        \end{itemize}
        \item Si la intersección tiene más de dos operandos, se agrupan según el tipo  (T, SV, ON) y luego se coordinan con la conjunción ``y''.
\end{itemize}

Ejemplo, el axioma de clase equivalente ``Pizza and (tieneCobertura some (CoberturaDePizza and (tienePicante some MuyPicante)))'' de la clase pizzaPicanteEquivalente, contiene dos intersecciones las cuales poseen los mismos componentes: T$+$SV, por lo que utilizamos el conector ``que'' para subordinar el segundo operando. Con este criterio, el axioma anterior sería ``pizza picante es una pizza \emph{que} tiene alguna cobertura de pizza \emph{que} tiene algún picante muy picante''.

En caso de asociar la intersección al conector \emph{y} sin tener en cuenta el contexto, se generaría una oración como la siguiente ``pizza picante es una pizza \emph{y} tiene alguna cobertura de pizza \emph{y} tiene algo muy picante'', dando lugar a ambigüedad, pues la segunda ``y'' podría cumplir la función de adición, interpretándose que la pizza tiene algo picante, en lugar de cumplir una función de subordinación, donde la interpretación correcta es que la cobertura de la pizza es picante. 

\subsection{Enumeración de propiedades, subclases e individuos}
Enumerar esta información es solo para anunciar la existencia de estos componentes, por lo que la sintaxis es simple.

Para las subclases se creó el patrón de la Gramática~\ref{gram:subclases}.
Para los individuos se creó un patrón similar al de las subclases.
Para las propiedades se creó el patrón de la Gramática~\ref{gram:propiedades}.

\begin{GrammarEnv}
\begin{grammar}
[(colon){$\rightarrow$}]
[(semicolon)$|$]
[(comma){}]
[(period){\vspace{0.3cm} \\}]
[(quote){\begin{bf}}{\end{bf}}]
[(nonterminal){$<$}{$>$}]
\fbox{\begin{minipage}{14cm}
<EnumerarSubclases> : <patron$_1$> ; <patron$_2$>.
<patron$_1$> : "existen las siguientes clases de" <topico> "$\colon$" <subclases>.
<patron$_2$> : <subclase> "es la única clase de" <topico>.
\end{minipage}}
\end{grammar}
\caption{Patrones para enumerar subclases.}\label{gram:subclases}
\end{GrammarEnv}

\begin{GrammarEnv}
\begin{grammar}
[(colon){$\rightarrow$}]
[(semicolon)$|$]
[(comma){}]
[(period){\vspace{0.3cm} \\}]
[(quote){\begin{bf}}{\end{bf}}]
[(nonterminal){$<$}{$>$}]
\fbox{\begin{minipage}{14cm}
<EnumerarPropiedades> : <patron$_1$> ; <patron$_2$> ; <patron$_3$>.
<patron$_1$> : <parteInicialPropiedad> <complementoPropiedad> <rangoPropiedad> .
<patron$_2$> : <parteInicialPropiedad> "como" <complementoPropiedad> "a" <rangoPropiedad> .
<patron$_3$> : <parteInicialPropiedad> <rangoPropiedad> .
\end{minipage}}
\end{grammar}
\caption{Patrones para enumerar propiedades.}\label{gram:propiedades}
\end{GrammarEnv}



\subsection{Expresiones de referencia}
El proceso de referenciación se ve facilitado por el uso de la Teoría de Centrado. 
Teniendo en cuenta que cada párrafo (y por lo tanto cada oración que lo compone) trata centralmente una única Clase o Individuo, no se intercalan tópicos como focos de atención, por lo que se reduce la posibilidad de ambigüedad. Las expresiones de referencia solo fueron implementadas para las Clases.

Para llevar a cabo la referenciación, se tuvo en cuenta el uso de pronombres demostrativos (``este/esta''), a veces, acompañados por el primer sustantivo (si posee) del nombre de la Clase. Por ejemplo: sean las oraciones ``Una Rosa es una pizza con nombre. Una Rosa tiene como cobertura a alguna cobertura de gorgonzola, mozzarella Y tomate.'' en la segunda oración en lugar de repetir ``una Rosa'', se reemplaza por una expresión de referencia ``Una Rosa es una pizza con nombre. \emph{esta} tiene como cobertura a alguna cobertura de gorgonzola, mozzarella Y tomate.''. Otro ejemplo en el que el pronombre es acompañado por el sustantivo de la Clase es: ``una cobertura de salsa de chile tabasco es una cobertura de salsa. \emph{Esta cobertura} tiene como picante algo muy picante.''

%Otro tipo de referencias, se da en el momento de explicar una clase. Cuando se crea una sección hablando acerca de una clase \emph{A}, esa sección no vuelve a crearse si existe una clase \emph{B} cuya descripción requiere que se explique la clase \emph{A}, sino que en la clase \emph{B} se hace referencia a la sección donde ya fue explicada la clase \emph{A}. Esto permite reutilizar las secciones y evita agregar información redundante en el texto.

\subsection{Agregación de sentencias}
La agregación de sentencias ocurre en dos lugares. Uno es durante el proceso de producción de una oración en el constructor \emph{N-Ary}, es decir dentro de una oración; el otro es durante el proceso de creación de párrafos, es decir, entre oraciones.

%de marco metodologico para la gnl
Basaremos la agregación en la conjunción por componentes compartidos~\cite{bernardos2003marco}. El objetivo es que los elementos en común aparezcan una sola vez, mediante elipsis del componente repetido, por ejemplo: sean las oraciones: ``la casa tiene color rojo'', ``la casa tiene color azul'' y ``la casa tiene color verde'', al comparar linealmente las oraciones, se elimina de las consecutivas oraciones la parte inicial que tengan en común. El resultado de esta agregación es: ``la casa tiene color rojo, azul y verde''.
    
\begin{itemize}
    \item {\bf Agregación en constructor NAry}: en las situaciones en las que este constructor se utiliza para enumerar sentencias, como el orden de los elementos enumerados no altera el significado de la oración, se optó por agrupar los elementos según su función (T, SV, ON). Para el grupo de sentencias T, se omitió información inicial de cada sentencia. Por ejemplo: las oraciones ``pizza con carne'', ``pizza picante'' y ``pizza no vegetariana'', se convierten en la oración ``pizza con carne, picante y no vegetariana''.
    
    Para los grupos SV y ON, se ordenaron por verbo, y cada verbo aparece una sola vez, acompañado por la enumeración de los complementos de cada elemento. Por ejemplo: las oraciones ``no tiene como cobertura alguna cobertura de carne'' y ``no tiene como cobertura alguna cobertura de pescado'' se convierten en ``no tiene como cobertura alguna cobertura de carne ni pescado''.
    
    \item {\bf Agregación entre sentencias}: esta agregación se aplica cuando se recorren las sentencias para armar un párrafo. Además de agrupar las sentencias por función (T, SV, ON), también se separan por constructor, obteniendo así 8 grupos: \emph{ObjectSomeValuesFrom, ObjectAllValuesFrom, intersection, union, ObjectHasValue, T, ON y SV}. De esta manera, para cada grupo, se realiza la agregación, de manera independiente entre ellos.
\end{itemize}

\subsection{Ejemplos verbalización de Axiomas}
Con el objetivo de mostrar el proceso de verbalización de forma intuitiva, se representó la composición de oraciones parciales de forma gráfica usando árboles. Se tomaron dos ejemplos de Axiomas de dos ontologías diferentes, uno en lenguaje español y el otro en inglés.

Para no sobrecargar la imagen, se omitió información de menor importancia, como el resultado de verbalizar una Entidad en los nodos hojas, ya que es el mismo nombre de la Entidad pero sin la nomenclatura CamelCase; y se omitió la información gramatical de cada palabra.

\subsubsection{Ejemplo verbalización lenguaje español}
El ejemplo se tomó de la ontología Pizza. El Axioma se corresponde al constructor \emph{owl:equivalentClass} sobre la clase \emph{PizzaPicanteEquivalente}. El contenido del axioma es el siguiente: 
\begin{verbatim}
EquivalentClasses(
    :PizzaPicanteEquivalente
    ObjectIntersectionOf(
        :Pizza
        ObjectSomeValuesFrom(
            :tieneCobertura
            ObjectIntersectionOf(
                :CoberturaDePizza
                ObjectSomeValuesFrom(
                    :tienePicante
                    :MuyPicante)
            )
        )
    )
)    
\end{verbatim}
La verbalización se muestra en la figura~\ref{fig:ejemplo_verb_espaniol}.

Los pasos del proceso se encuentran enumerados desde el 1 al 6. La verbalización comienza en la parte más profunda del árbol, y a medida que se generan las oraciones parciales se retornan al nivel superior del árbol para continuar la composición.

Las composiciones se resuelven a través de las Gramáticas de la sección~\ref{sec:verbalizacion_constructores}. Por ejemplo, en el paso Nº1 utiliza patron$_1$ de la Gramática~\ref{gram:quant_obj_rest}, donde:
\begin{itemize}
    \item $<$verboPropiedad$>$ es ``tiene''.
    \item $<$complementoPropiedad$>$ es ``picante''.
    \item $<$enlace$>$ es ``algún''.
    \item $<$clases$>$ es ``muy picante''.
\end{itemize}

En el paso Nº2, como es una intersección de dos operandos, utiliza la subordinación de SV respecto a T.

De esta manera, en cada nivel busca utilizar un patrón que se ajuste a los componentes recibidos.

\begin{figure}
    \centering
    \includegraphics[width=\textwidth]{img/generacion_documento/verbalizacion_equivalentClass_spanish.pdf}
    \caption{Ejemplo gráfico de verbalización en lenguaje español.}
    \label{fig:ejemplo_verb_espaniol}
\end{figure}

\subsubsection{Axioma lenguaje inglés}
El ejemplo se tomó de la ontología Wine. El Axioma se corresponde al constructor \emph{owl:equivalentClass} sobre la clase \emph{CheninBlanc}. El contenido del axioma es el siguiente: 

\begin{verbatim}
EquivalentClasses(
    :CheninBlanc
    ObjectIntersectionOf(
        :Wine
        ObjectHasValue(
            :madeFromGrape
            :CheninBlancGrape)
        ObjectMaxCardinality(
        1
        :madeFromGrape)
    )
)
\end{verbatim}
La verbalización se muestra en la figura~\ref{fig:ejemplo_verb_ingles}.

Los pasos del proceso se encuentran enumerados desde el 1 al 4. Las gramáticas utilizadas son similares a las explicadas para lenguaje español, con la diferencia que los enlaces han sido traducidos a inglés.

Resulta conveniente explicar el paso Nº3 particularmente. El resultado de este paso es consecuencia de la intersección entre ``Wine'' y el resultado del paso 1 y 2. 
Como es una intersección de tres componentes, se procede a agrupar los componentes según su tipo, para luego unirlos a través de la conjunción ``y''. Dado que el operador \emph{hasValue} y \emph{minCardinality} retornan componentes de tipo SV, son agrupados juntos en una oración. Luego de agruparlos, se procede a realizar el proceso de agregación entre ambas, a través del cual se elimina el verbo ``made'' de la oración parcial construida en el paso 1. 

Por otro lado, la Clase Wine retorna un tipo T, por lo que queda aislado en una oración independiente. Estas agrupaciones resultan en el texto del paso Nº3.

La verbalización de este axioma da como resultado dos oraciones. Como ambas oraciones hacen referencia a la misma Clase, en el paso Nº4 ocurre un reemplazo del nombre de la Clase por una Expresión de Referencia en la segunda oración. 

\begin{figure}
    \centering
    \includegraphics[width=\textwidth]{img/generacion_documento/verbalizacion_equivalentClass_english.pdf}
    \caption{Ejemplo gráfico de verbalización en lenguaje inglés.}
    \label{fig:ejemplo_verb_ingles}
\end{figure}

\subsection{Diagrama de clases}
En la figura \ref{fig:diagrama_clases_microplanificador} se puede ver el diagrama de clases asociado al módulo de Micro Planificación. 

La clase \emph{Statement} es la encargada de realizar la verbalización de los constructores de Axiomas y las enumeraciones. Un objeto \emph{Statement} puede generar más de una oración, ya que cada \emph{Statement} trata un solo tópico y un solo tipo de constructor, pero pueden haber varios axiomas que utilicen el mismo tipo de constructor, por lo que todos esos axiomas estarán en la misma \emph{Statement}.

La clase \emph{StatementComponent} representa la oraciones parciales. Se encarga de componer oraciones parciales y generar una nueva oración parcial. En el caso de un constructor de Entidad, se encarga de generar una oración parcial compuesta por el nombre de la Entidad.

La clase \emph{Word} representa las palabras usadas durante la verbalización. Contiene la función gramatical de la palabra, y los algoritmos necesarios para convertir una palabra en plural, un número en palabra (1 a uno), cambiar el género de una palabra, retornar el artículo correspondiente a un sustantivo, entre otros.

Las subclases de \emph{OWLRestriction} tienen la capacidad de verbalizar los constructores de Expresiones de Clases y de generar las \emph{StatementComponent}.

\begin{figure}[H]
    \centering
    \includegraphics{img/generacion_documento/diagrama_clases_microplanificador.pdf}
    \caption{Diagrama de clases del Micro Planificador.}
    \label{fig:diagrama_clases_microplanificador}
\end{figure}


\section{Implementación Micro Planning}
Para facilitar la implementación de los algoritmos de Agregación y Expresiones de Referencia, decidimos implementar el contenido de las oraciones con listas de palabras en lugar de cadenas de texto, facilitando su manipulación y recorrido. De esta manera, se creó la interfaz \emph{Word} con sus respectivas subclases \emph{WordEnglish} y \emph{WordSpanish}. 
La clase \emph{WordSpanish} utiliza la librería \emph{javaGramatica}\footnote{\url{https://www.proinf.net/permalink/gramatica_numero_genero_y_acentuacion}}, para implementar algunos métodos del idioma español, como pluralizar, cambio de género, número, etc. Para el idioma inglés estos métodos fueron implementados manualmente.

Cada vez que se crea una Word, debe especificarse su palabra y el tipo de función que cumple la palabra. Para identificar el tipo de una palabra, se utilizó el software\texttt{Stanford POS Tagger}\footnote{https://nlp.stanford.edu/software/tagger.html}, el cual permite etiquetar palabras de inglés, español y otros idiomas. Este software utiliza una nomenclatura específica para etiquetar las palabras, por lo que utilizamos esa nomenclatura y la implementamos a través de Variables de Clase creadas en la clase Word.

Los objetos \emph{Word} son creados por cada \emph{StatementComponent}. Una \emph{StatementComponent} se crea durante el proceso de verbalización de los constructores de Expresiones de Clase y de Entidades. Como los constructores del nivel más bajo son de Entidades, es en ellas donde se decide qué tipo de objeto se instancia según el idioma, es decir \emph{StatementComponentEnglish} o \emph{StatementComponentSpanish}. 

En el nivel de los constructores de Expresiones de Clase se encuentran las clases que heredan de la interfaz \emph{OWLRestriction}. Esta interfaz acopla los atributos que requieren los constructores, como los individuos, las clases, las propiedades y cardinalidades.

Las clases que implementan \emph{OWLRestriction} componen y generan las \emph{StatementComponent}. Como ejemplo, se muestra en la Figura~\ref{fig:clase_OWLObjectComplementOf} la implementación de la clase \emph{OWLObjectComplementOf} con los métodos para implementar la verbalización del constructor en español.

Se puede ver que el método \emph{generateStatementSpanish()} recibe como párametro un objeto de clase \emph{TextCotext}. Esta clase tiene el tópico al que hace referencia la oración que se está componiendo y el lenguaje de la oración, para ser utilizados, por ejemplo, en la generación de Expresiones de Referencia.

Por último la interfaz \emph{Statement} tiene los atributos necesarios para que sus implementaciones \emph{StatementEnglish} y \emph{StatementSpanish} puedan verbalizar los constructores de Axiomas y de unir todas las oraciones. 

Las \emph{Statement} utilizan la Agregación y las Expresiones de Referencia cuando generan varias oraciones. Estas tareas se implementaron en las clases \emph{StatementComponent} (teniendo en cuenta sus implementaciones) y \emph{OWLClass} respectivamente. 

\begin{figure}
\begin{minted}[autogobble,linenos]{java}
public class OWLObjectComplementOf extends OWLRestriction{

    public OWLObjectComplementOf(String type, String lang) {
        super(type, lang);
    }

@Override
    protected StatementComponent generateStatementSpanish(TextCotext c) {
        LinkedList<StatementComponent> stmsClasses = new LinkedList<>();
        stmsClasses.add(classes.get(0).generateStatement(c));
        LinkedList<Word> resList = new LinkedList<>();
        LinkedList<Word> enlace = getEnlaceSpanish(stmsClasses.get(0));
        resList.addAll(enlace);
        resList.addAll(stmsClasses.get(0).getListWords());
        StatementComponent stm = new StatementComponentSpanish(c, resList, "ON", type);
        stm.setComplementList(resList);
        return stm;
    }
    
    private LinkedList<Word> getEnlaceSpanish(StatementComponent stm) {
        LinkedList<Word> enlace = new LinkedList<>();
        if (stm.getType().equals("T")) {
            WordSpanish lo = new WordSpanish("LO", WordSpanish.TYPE_PRONOUN);
            WordSpanish opuesto = new WordSpanish("OPUESTO", WordSpanish.TYPE_ADJETIVE);
            WordSpanish de = new WordSpanish("DE", WordSpanish.TYPE_PREPOSITION);
            enlace.add(lo);
            enlace.add(opuesto);
            enlace.add(de);
        } else if (stm.getType().equals("SV")) {
            WordSpanish no = new WordSpanish("NO", WordSpanish.TYPE_ADV_NEG);
            enlace.add(no);
        } else if (stm.getType().equals("ON")) {
            WordSpanish ni = new WordSpanish("NI", WordSpanish.TYPE_CONJUNTION_COORD);
            enlace.add(ni);
        }else {
            WordSpanish no = new WordSpanish("EXCEPTO", WordSpanish.TYPE_ADV_NEG);
            enlace.add(no);
        }
        return enlace;
    }
}
\end{minted}
\caption{Implementación de los métodos para verbalizar el constructor \emph{OWLObjectComplementOf} en español.}
\label{fig:clase_OWLObjectComplementOf}
\end{figure}

\section{Diseño del realizador}
%La arquitectura de este trabajo relegó la tarea de definir la estructura de las oraciones y llevar acabo su realización lingüística a la etapa de Micro Planning.
En esta etapa queda la tarea de definir la estructura final del documento. 

Alcanzaremos el Documento Final, a través del refinamiento del Documento Inicial creado en el Macro Planning.

El Documento Inicial tiene una estructura en la que todos los tópicos tienen su propia sección, por lo que no existe más información que haga posible expandir la estructura. Por este motivo, nos enfocaremos en la idea de reducir la estructura del documento convirtiendo secciones en párrafos, para que se haga más compacto, tratando de mejorar la estética y el proceso de lectura, sin perjudicar la coherencia y la segmentación de la información.

Para llevar a cabo el refinamiento, se buscó medir la información a verbalizar, para saber cuándo es posible convertir una sección. Para esto se propuso como métrica principal a la cantidad de oraciones presentes en la descripción de un tópico.

\subsection{Criterio de reducción de Secciones}
Una sección agrupa uno o más párrafos o sub-secciones, por lo que en una sección puede existir más de un tópico, con la condición de que esos tópicos tengan una relación, ya sea de hermanos o de hijos. Estas relaciones se obtienen de la Estructura de Árbol del Organizador de Información. 

Para decidir si un tópico es planificado como una Sección, se utilizan los siguientes criterios:
\begin{itemize}
    \item Debe poseer cinco o más oraciones.
    \item Debe poseer alguna Subsección con cinco o más oraciones.
    \item En cualquier otro caso será planificado como un Párrafo.
\end{itemize}

Si la cantidad de oraciones requeridas para ser considerado una sección, tiende a cero, la planificación del documento será similar al Documento Inicial; mientras que si la cantidad de oraciones tiende a infinito, la planificación perderá niveles de jerarquía, y únicamente contemplará como secciones a los tópicos del primer nivel. Ninguno de los dos extremos es conveniente: o se pierde coherencia, o se obtiene un texto con demasiados niveles de secciones, que resulta antinatural. Sin embargo, un documento en el que predominan las sub-secciones (aunque tengan poca información), ayuda a los humanos a reconocer y establecer relaciones entre los tópicos, siendo el objetivo principal no desaprovechar la semántica del dominio modelado, y teniendo como intención principal que el lector comprenda el dominio modelado en la ontología.

\section{Implementación del realizador}
El realizador está embebido dentro de la clase Section, en forma de métodos que implementan los criterios vistos para la definición del documento. En la figura~\ref{fig:clase_section} se presenta un pseudocódigo de la clase Section.

\begin{figure}
\begin{minted}[autogobble,linenos]{java}
public class Section {

    private LinkedList<OWLIntClass> topics;
    private Title title;
    private LinkedList<Paragraph> paragraphs;
    private LinkedList<Section> subSections;
    private String language;

    public Section(OWLIntClass topic, String lang) {
        init(lang);
        topics.add(topic);
        title = new Title(topic, lang);
        if (topic.getType().equals("class") ||
        topic.getType().equals("individual")) {
            //crear el Paragraph para la informacion de la seccion
            Paragraph inf = new Paragraph(topic, lang);
            paragraphs.add(inf);
        } 
    }
    
    public String generateText() {
    String textF = "";
    textF = titulo de la Section
    textF += contenido de los Parrafos de la Section
    for (Section s: subSections){
        if (s no cumple los criterios para verbalizar como sección){
            textF += contenido de s en forma de párrafo
        }
    }
    /*La realización de las subsecciones se separa en dos
    recorridos distintos para que todos los párrafos estén
    juntos, y luego aparezcan todas las secciones.*/
    for (Section s: subSections){
        if (s cumple los criterios para verbalizar como sección){
            textF += s.generateText();
        }
    }
    
    return textF;
    }
    
}
\end{minted}
\caption{Definición en Java de la Clase \texttt{Section}.}
\label{fig:clase_section}
\end{figure}


\section{Casos de estudio}
En el capítulo de Organización de Información analizamos cómo quedan organizados los tópicos de las ontologías Pizza y Wine, por lo que continuaremos con estas ontologías de ejemplo, analizando la realización del texto generado. Para identificar el formato del texto, a las secciones se le incluye el número de sección a la izquierda del título de la sección, y cada párrafo comienza con sangría.

\subsection{Generación documento ontología Pizza}
Ya que el texto completo resulta muy extenso, solo pondremos los fragmentos más significativos del Documento Final. En la figura~\ref{fig:doc_final_pizza} se presentan los fragmentos a analizar.

\begin{figure}
\fbox{
\begin{minipage}{14cm}
\setlength{\parindent}{1em}

\noindent 1 Pizza

una pizza es una comida. una pizza tiene base de pizza y tiene cobertura de pizza. Esta tiene como base a alguna base de pizza. Existen las siguientes clases de pizzas: pizza con carne, picante, no vegetariana, con nombre, \dots, con queso y vegetariana. una pizza es disjunta de base de pizza, cobertura de pizza y helado.

Pizza con carne: una pizza con carne es una pizza que tiene como cobertura a alguna cobertura de carne. 
\par \dots Otras pizzas realizadas como párrafos.

\noindent 1.1 Pizza con nombre

una pizza con nombre es una pizza. Existen las siguientes clases de pizzas con nombre: margherita, frutti di mare, \dots, veneziana y napoletana.

Margherita: una margherita es una pizza con nombre. Esta tiene como cobertura a alguna cobertura de mozzarella y tomate. También, margherita tiene exclusivamente las siguientes coberturas: cobertura de mozzarella o de tomate. una margherita es disjunta de american, american hot, \dots, soho y veneziana.

\dots Otras pizzas con nombre realizadas como párrafos.

\noindent 1.2 Ingredientes
	
\noindent 1.2.1 Coberturas
	
\noindent 1.2.1.1 Cobertura de pizza

una cobertura de pizza es una comida. una cobertura de pizza es cobertura de una pizza. Existen las siguientes clases de coberturas de pizza: \dots

\noindent 1.2.2 Bases

\noindent 1.2.2.1 Base de pizza

una base de pizza es una comida. una base de pizza es base de una pizza. Existen las siguientes clases de bases de pizza: base gruesa y delgada y crujiente.
\end{minipage}
}
\caption{Fragmentos del Documento Final generado a partir de la ontología Pizza.}
\label{fig:doc_final_pizza}
\end{figure}
Para visualizar la estructura total del documento, se imprimió el índice de las secciones, el cual se muestra en la Figura~\ref{fig:indice_secciones_pizza}. Se puede apreciar que la estructura del documento comparte algunas similitudes con el Documento Inicial, manteniendo la coherencia global, pero disminuyó la cantidad de secciones, ya que muchas de ellas se convirtieron en párrafos. Aún así, prevalecen títulos de secciones que agregan información semántica aunque no aporten información en forma de texto, tales como Ingredientes, Coberturas y Bases.

\begin{figure}
\begin{multicols}{2}
\begin{figure}[H]
\dirtree{%
.1 Pizza.
.2 Pizza con nombre.
.2 Ingredientes.
.3 Coberturas.
.4 Cobertura de pizza.
.5 Cobertura de verduras.
.6 Cobertura de pimiento.
.6 Cobertura de tomate.
.5 Cobertura de hierbas.
.5 Cobertura de carne.
.5 Cobertura de queso.
.5 Cobertura de pescado.
.3 Bases.
.4 Base de pizza.
.5 Base gruesa.
.5 Base delgada y crujiente.
}
\end{figure}

\begin{figure}[H]
\dirtree{%
.1 Otras secciones.
.2 Domain concept.
.3 Comida.
.2 Value partition.
.3 Picante.
}
\end{figure}
\end{multicols}
\caption{Índice de secciones del Documento Final generado para la ontología Pizza.}
\label{fig:indice_secciones_pizza}
\end{figure}


\subsection{Generación documento ontología Wine}
Al igual que el caso de estudio anterior, solo incluiremos algunos fragmentos del documento generado, los cuales se pueden ver en la Figura~\ref{fig:doc_final_wine}.

En la Figura~\ref{fig:indice_secciones_wine} se visualiza el índice del documento. El comportamiento es similar al obtenido en la verbalización de la ontología Pizza, comparando la Figura~\ref{fig:indice_secciones_wine} con la Figura~\ref{fig:caso_estudio_wine}, se observa que la estructura resulta mucho más compacta.


\begin{figure}
\fbox{
\begin{minipage}{14cm}
\setlength{\parindent}{1em}
\noindent 1 Wine

a wine is a potable liquid. a wine has wine sugar as sugar, has wine flavor as flavor, has wine body as body, has wine color as color, has wine descriptor and made from grape a wine grape. This located in SOME region. Also, wine has only winery maker. Also, wine has exactly one maker, made at least one wine grape, wine body, wine color, wine flavor and wine sugar. There are the following kinds of wines: italian wine, \dots, late harvest and alsatian wine. 

Full bodied wine: a full bodied wine is a wine that has full body.

\noindent 1.1 Italian wine

a italian wine is a wine that located in italian region. chianti is the only kind of wine. 

Chianti: a chianti is a italian wine. This has only light or medium body. Also, chianti has red color, moderate flavor and dry sugar. located in chianti region. made from grape sangiovese grape. chianti classico is a type of chianti. 

Chianti classico: a chianti classico has mc guinnesso maker and medium body.

\noindent 1.14 Wines descriptor

Wine sugar:	a wine sugar is a wine taste. a wine sugar is a sweet, off dry or dry. there are the wines sugar: dry, off dry and sweet. 


Wine color: a wine color is a wine descriptor. a wine color is a rose, red or white. there are the wines color: white, red and rose. 


Wine flavor: a wine flavor is a wine taste. a wine flavor is a moderate, delicate or strong. there are the wines flavor: moderate, strong and delicate. 

Wine body: a wine body is a wine taste. a wine body is a light, medium or full. there are the wines body: medium, full and light. 

\end{minipage}
}
\caption{Fragmentos del Documento Final generado a partir de la ontología Wine.}
\label{fig:doc_final_wine}
\end{figure}


\begin{figure}
\begin{multicols}{2}
{\small
\begin{figure}[H]
\dirtree{%
.1 Wine.
.2 Italian wine.
.2 Loire.
.2 Table wine.
.2 Pinot noir.
.2 Dessert wine.
.3 Sweet riesling.
.2 Bordeaux.
.3 Medoc.
.2 Riesling.
.2 Red wine.
.3 Red burgundy.
.2 Cabernet sauvignon.
.2 Semillon or sauvignon blanc.
.3 Sauvignon blanc.
.2 White wine.
.3 White burgundy.
.2 Zinfandel.
.2 Chardonnay.
.2 Wines descriptor.
.3 Wine sugar.
.3 Wine color.
.3 Wine flavor.
.3 Wine body.
}
\end{figure}}  

\begin{figure}[H]
\dirtree{%
.1 Otras secciones.
.1 Region.
.1 Vintage year.
.1 Wine descriptor.
.1 Non consumable thing.
.1 Vintage.
.1 Fruit.
.1 Winery.
.1 Consumable thing.
.2 Meal course.
.3 Foods.
.4 Edible thing.
.5 Seafood.
.5 Pasta.
}
\end{figure}

\end{multicols}
\caption{Resultado del organizador de información con la ontología Wine.}
\label{fig:indice_secciones_wine}
\end{figure}

\section{Conclusiones}



\chapter{Conclusiones}
En el desarrollo de esta tesis pudimos diseñar e implementar un Sistema de Generación de Lenguaje Natural, orientado a generar un texto organizado a partir de una ontología de la Web Semántica. Para desarrollar tal sistema se propusieron dos problemas fundamentales: el de organización de la información y el de generar el documento de texto. 

Para abordar el problema de la organización de la información, se tomó en cuenta que los textos poseen una estructura jerárquica en el que sus tópicos presentes están semánticamente relacionados, al igual que las ontologías. Para maximizar la relación semántica entre los tópicos, se usó como base la coherencia global, aplicando los fundamentos de la Teoría de Veins. Se propuso una solución basada en el análisis del grafo subyacente a la ontología, midiendo las relaciones entre sus nodos. De esta manera se obtuvieron las Entidades de la ontología que resultaban más sobresalientes, con el fin de usarlas como hilo conductoras de los temas del texto.
Esta solución se vio motivada por la característica que poseen las ontologías de la Web Semántica, de ser grafos libres de escala. Los resultados obtenidos fueron satisfactorios para los casos de prueba presentados, retornando estructuras organizadas que se correspondían con los resultados esperados. Sin embargo queda pendiente poder analizar la solución con mayor diversidad de ontologías, variando la densidad de sus relaciones, pues la solución propuesta depende ampliamente de la cantidad de \emph{ObjectProperties} y del contenido de sus dominios.

Para el sistema de generación de texto, se dividió el problema en tres sub-problemas, basados en los procesos conocidos del Procesamiento del Lenguaje Natural. Estos subproblemas fueron diseñar e implementar el \emph{Macroplanning}, \emph{Microplanning} y la \emph{Realización}. 

Para comenzar, se desarrolló el módulo \emph{Macroplanning}, encargado de brindar la estructura jerárquica del texto. Dado que se integró el \emph{Macroplanning} con el Organizador de Información, su implementación fue trivial.

Luego se procedió a verbalizar los axiomas de la ontología en el \emph{Microplanning}, desarrollando la gramática del lenguaje humano e implementando los algoritmos de generación de texto adecuados, para reducir la redundancia y mejorar la fluidez del texto. 

Por último se resolvió el problema de la Realización del texto, modificando la estructura inicial obtenida del \emph{Macroplanning}, para reducir los niveles de la jerarquía del texto donde era adecuado. 

Como se mostró en los casos de prueba, la solución basada en el desarrollo de estos módulos resultó en textos legibles y organizados. Si bien validar textos puede resultar subjetivo, podemos apreciar que existe suficiente relación semántica en el orden de los temas que se presentan, tanto entre las Secciones como entre las Oraciones, por lo que cumplen con cierta Coherencia Global y Local.

En cuanto a la sintaxis, se demostró que pueden existir mejores formas de presentar la información, que depende de cada caso particular.

Aún resulta conveniente explorar otras técnicas para desarrollar las gramáticas y la composición de las oraciones. Desarrollar las gramáticas manualmente es un trabajo costoso, y debido a la naturaleza recursiva de las construcciones de los axiomas y de la semántica de sus constructores, es difícil mantener el control de la oración compuesta, tanto a nivel sintáctico como semántico. Queda pendiente probar el sistema con otras ontologías que presenten mayor complejidad en sus axiomas, con el fin de desarrollar nuevas gramáticas que abarquen más casos de prueba.

Más allá de probar el sistema con nuevos casos de prueba, para reconocer las ventajas y desventajas de la solución propuesta, aún hay ciertos factores a determinar, para ayudar a mejorar la estética del texto, que quedan fuera del alcance de esta tesis, por ejemplo:
\begin{itemize}
    \item Decidir cuándo es conveniente utilizar texto en formato lista en lugar de prosa, o en algún otro formato como tablas comparativas.
    \item Decidir cuándo es conveniente utilizar expresiones de referencia o elipsis.
    \item Establecer criterios más sofisticados para modificar la estructura del texto en la etapa de Realización. 
\end{itemize}


\bibliographystyle{abbrv}
\bibliography{bibliografia}

\chapter*{Apéndice}
\appendix
%\markboth{Appendices}{}
\addcontentsline{toc}{chapter}{Apéndice}
\renewcommand{\thesection}{A.\arabic{section}}

\section{Traza de la generación de la estructura inicial para la ontología Pizza}
\label{apx:traza_pizza}

\centering
\setlength{\parindent}{1em}
\begin{verbatim}

Comienza la clasificación de las Entidades
para seleccionar los temas principales:
Participación de cada clase en la ontología
(cantidad de veces que aparece en un dominio):
    Pizza: 2
    BaseDePizza: 1
    Comida: 1
    CoberturaDePizza: 1
Suma total de participación: 5.0
Promedio de participación (suma total de participación
divido por la cantidad de clases que 
participaron de la suma):
    5.0/4 = 1.25
Clases que superan el promedio y por lo tanto
son consideradas temas principales:
[Pizza] -> porcentaje de participación en 
los dominios de la ontología: 40.0%
Temas principales reconocidos:[Pizza]

Comienza a crear los grupos para la Clase: Pizza
Crea subnivel
Comienza a crear el grupo para: PizzaConCarne
Comienza a crear el grupo para: PizzaPicante
Comienza a crear el grupo para: PizzaNoVegetariana
Comienza a crear el grupo para: PizzaConNombre
Crea subnivel
...
Sale del subnivel actual.
...
Comienza a crear el grupo para: PizzaVegetariana
Comienza a crear el grupo para: tieneIngrediente
Crea subnivel
Comienza a crear el grupo para: tieneCobertura
Crea subnivel
Comienza a crear el grupo para: CoberturaDePizza
Crea subnivel
...
Sale del subnivel actual.
Comienza a crear el grupo para: tieneBase
Crea subnivel
Comienza a crear el grupo para: BaseDePizza
Crea subnivel
Comienza a crear el grupo para: BaseGruesa
Comienza a crear el grupo para: BaseDelgadaYCrujiente
Sale del subnivel actual.
Sale del subnivel actual.
Sale del subnivel actual.
Sale del subnivel actual.
Finalizó la creación de la Sección rincipal.

Las siguientes Entidades no poseen su propia Sección,
por lo que serán agregadas a Otras Secciones: 
Comienza a crear el grupo para: DomainConcept
Crea subnivel
Comienza a crear el grupo para: Comida
...
Comienza a crear el grupo para: Pais
...
Sale del subnivel actual.
Comienza a crear el grupo para: ValuePartition
Crea subnivel
Comienza a crear el grupo para: Picante
Crea subnivel
Comienza a crear el grupo para: PocoPicante
Comienza a crear el grupo para: AlgoPicante
Comienza a crear el grupo para: MuyPicante
Sale del subnivel actual.
Sale del subnivel actual.
Finalizó la creación de Otras Secciones
\end{verbatim}

\section{Implementación Organizador de Información}
\label{apx:impl_org_inf}

\subsection{Clase \texttt{OntologyManager}}
\label{sec:clase_ontologymanager}

\begin{minted}[autogobble,linenos]{java}
public class OntologyManager {

    /*nameOf guarda el valor 0 o 1 para reconocer de donde tiene que
    extraer el nombre de las entidades*/
    private static int nameOf; // 0 -> iri, 1 -> label.
    /*lang setea el valor del lenguaje para buscar los nombres
    de las entidades en los labels correctos*/
    private static String lang; // "en" -> ingles, "es" -> español.

  public static Ontology createOntology
            (OWLOntology ontology, int nameOfClass, String language) {
        nameOf = nameOfClass;
        lang = language;
        Ontology onto = new Ontology();
        
        //Obtiene el título de la ontología desde la etiqueta title.
        String title = getTitleOntology();
        onto.setTitle(title);
        
        //crear clases
        OWLClass rootClass = createClasses();

        //crear los dataProperty
        OWLDataProperty rootDataProp = createDataProperty();
        
        //crear properties
        OWLObjectProperty rootProp = createObjectProperty();
        //setea las objectProperty equivalentes, inversas y disjuntas.
        setListsToProperties();
        //setea el dominio y rango de las objectProperty
        setObjectPropertyDomainRange();
        //setea el dominio y rango de las dataProperty
        setDataPropertyDomain();
        
        //crear individuals
        createIndividuals();
        
        /*crear las clases anónimas para cada clase en onto.
        Setea listas para guardar los axiomas. Para cada clase en onto
        setea sus clases equivalentes, disjuntas y superClases.*/
        /*createAnonymousClasses debe ser llamado despues de haber
        creado las clases nombradas, las properties y los individuos.*/
        createAnonymousClasses();
        
        //agrega los componentes creados al objeto onto.
        addComponentsToOnto();
        
        //precomputeValues asocia a cada clase en onto una lista 
        //de propiedades en las que participa como dominio.
        onto.precomputeValues();
        return onto;
    }
}
\end{minted}

\subsection{Clase \texttt{Ontology}}
\label{sec:clase_ontology}
\begin{minted}[autogobble,linenos]{java}
public class Ontology {

    private String title;
    //guardan las Entidades de nivel mas alto en la jerarquía.
    private OWLClass rootClass;
    private OWLObjectProperty rootObjectProperty;
    private OWLDataProperty rootDataProp;

    //los siguientes hashmaps asocian IRIs a sus respectivas Entidades.
    private HashMap<IRI, OWLClass> classes;
    private HashMap<IRI, OWLObjectProperty> properties;
    private HashMap<IRI, OWLObjectProperty> anonProperties;
    private HashMap<IRI, OWLDataProperty> dataProperties;
    private HashMap<IRI, OWLIndividual> individuals;
    
    //Las siguientes listas guardan las Entidades presentes en la 
    //ontologia
    private LinkedList<OWLClass> listClasses;
    private LinkedList<OWLObjectProperty> listProps;
    private LinkedList<OWLObjectProperty> anonListProps;
    private LinkedList<OWLDataProperty> listDataProps;
    private LinkedList<OWLIndividual> listIndividuals;

    /*hashmaps para guardar que clases participan en el dominio
    de que propiedades*/
    HashMap<OWLClass, LinkedList<OWLObjectProperty>> classOnPropDomain;
    /*hashmaps para guardar que clases participan en el rango
    de que propiedades*/
    HashMap<OWLClass, LinkedList<OWLObjectProperty>> classOnPropRange;

    /*Retorna una lista de ObjectProperties que sean superProperties
    en comun de las properties de la lista prop.*/
    public LinkedList<OWLObjectProperty>
    getCommonSuperProperties(LinkedList<OWLObjectProperty> prop) {}

    /*Dada la clase cls, recorre las properties donde participa como dominio
    y busca las superProperties de mas alto nivel que encuentre 
    y las retorna. */
     public LinkedList<OWLObjectProperty>
     getCommonSuperPropertiesFromClass(OWLClass cls) {}

    /*Retorna el contenido necesario para crear un nuevo grupo
    de entidades a partir de la clase cls. Util para
    el modulo ContentGrouping */
    public LinkedList<OWLIntClass> 
    createTopicsGroupsFromClass(OWLClass cls) {
        LinkedList<OWLIntClass> res = new LinkedList<>();
        res.addAll(getCommonSuperPropertiesFromClass(cls));
        res.addAll(cls.getSubClass());
        res.addAll(cls.getIndividuals());
        return res;
    }
}
\end{minted}


\subsection{Clase \texttt{ContentClasification}}
\label{sec:clase_content_clasif}

\begin{minted}[autogobble,linenos]{java}
public class ContentClasification {

    /* Es el método principal que retorna las entidades
    más relevantes para agregar en el primer nivel de
    la jerarquía del Árbol de Entidades. */
    public static HashMap<OWLIntClass, LinkedList<OWLObjectProperty>>
    getMainContent(Ontology ontology) {
        
        cls2Cant = getClassesToCantProperties(ontology);
        mainClasses = getMainClasses(cls2Cant);

        /*insertar en un nuevo hashmap cls2SuperProp las clases de mainClasses
        asociadas a las superProperties en común que haya entre las properties 
        donde participa como dominio*/
        for (OWLClass mainClass : mainClasses) {
            cls2SuperProp.put(mainClass,
            ontology.getCommonSuperPropertiesFromClass(mainClass));
        }
        
        /*Si existe una Clase c y alguna SuperClase de c, 
        entonces c se elimina por ser mas específica*/
        cls2SuperProp = removeSubClass(cls2SuperProp);

        /*si el hashmap es vacio, entonces se agregan todas las
        clases sin objproperties*/
        if (cls2SuperProp.isEmpty()) {
            for (OWLClass cls : ontology.getClasses()) {
                cls2SuperProp.put(cls, new LinkedList<>());
            }
        }
        return cls2SuperProp;
    }
    
    /*Devuelve una Lista de HashMaps desde Clases a Integers. Cada
     Hashmap tiene una sola entrada, que va desde una Clase a la
     cantidad de properties en las que participa como dominio.*/
    private static LinkedList<HashMap<OWLClass, Integer>>
    getClassesToCantProperties(Ontology ontology) {}
    
    /*Devuelve una lista de Clases. Las Clases retornadas son las
     que participan en mas cantidad de properties, y que superan 
     el promedio calculado como la suma de todas las properties
     dividido entre la cantidad de clases.*/
    private static LinkedList<OWLClass>
    getMainClasses(LinkedList<HashMap<OWLClass, Integer>> cls) {}
}

\end{minted}


\subsection{Clase \texttt{ContentGrouping}}
\label{sec:clase_content_grouping}

\begin{minted}[autogobble,linenos]{java}
public class ContentGrouping {

    /*A partir de la clase cls y la rama branch a la que pertenece,
    se genera un nuevo grupo de Entidades, y se verifica que cada
    Entidad aún no pertenezca a la rama*/
    public static LinkedList<OWLIntClass>
    generateNewTopics(Ontology ontology, OWLClass cls, 
                LinkedList<OWLIntClass> branch) {
        LinkedList<OWLIntClass> tops = ontology.createTopicsGroupsFromClass(cls);
        for (OWLIntClass r : tops) {
            if (!r.getType().equals("property")) {
                    if (!branch.contains(r)) {
                        res.add(r);
                    }
            }
        }
        for (OWLIntClass r : tops) {
            if (r.getType().equals("property")) {
                checkProperties((OWLObjectProperty) r, res, branch);
            }
        }
        return res;
    }

    /*A partir de una property prop y la clase dom que pertenece
    a su dominio, busca crear un nuevo grupo de Entidades.
    Primero busca subProperties de prop. Si no tiene subProperties, 
    entonces intenta agregar el Rango de prop.*/
    public static LinkedList<OWLIntClass> 
    generateNewTopics(Ontology ontology, OWLObjectProperty prop,
                    OWLClass dom) {
        
        /*obtiene las subProperties de prop*/
        LinkedList<OWLIntClass> newTopics = prop.getSubPropWithDomain(dom);
        
        if (newTopics.isEmpty()) {
            newTopics = prop.getRange();
        }
        return newTopics;
    }
    
    /*A partir de prop busca agregar a ella o una superProperty
    de ella en res. Agrega a prop si tiene una superProperty en branch.
    Si no puede agregar a prop busca agregar un ancestro de prop que 
    tenga una superProperty en branch.*/
    private static boolean checkProperties(OWLObjectProperty prop,
    LinkedList<OWLIntClass> res, LinkedList<OWLIntClass> branch) {}
}
\end{minted}

\subsection{Clase \texttt{GeneratorTreeManager}}
\label{sec:clase_generator_tree}

\begin{minted}[autogobble,linenos]{java}
public class GeneratorTreeManager {

    public static LinkedList<OWLIntClass> getText(Ontology onto) {

        HashMap<OWLIntClass, LinkedList<OWLObjectProperty>> mainClass =
        ContentClasification.getMainContent(onto);
        
        /*agregar las clases de mainClass al primer nivel del árbol. 
        mainEntities representa el primer nivel del árbol 
        (las entidades mas relevantes)*/
        mainEntities.add(mainClass);
        
        /*para cada clase de mainClass crear una nueva rama y agregar 
        sus Nuevos grupos de Entidades*/
        for (Map.Entry<OWLIntClass, LinkedList<OWLObjectProperty>> entrySet : 
                mainClass.entrySet()) {
            /*navigation mantiene las clases del primer nivel y
            las Entidades de la rama que se está creando*/
            navigation.add(mainClass);
            OWLClass mClass = entrySet.getKey();
            LinkedList<OWLIntClass> newEntities =
            ContentGrouping.generateNewTopics(onto, mClass, null);
            addBranchFromEntity(newBranch, mClass, newEntities, navigation,
            onto);
            mainEntities.addNewBranchToEntity(mClass, newBranch);
            navigation.clear();
            
        }
        /*agregar informacion marginada*/
        mainEntities.addNewBranch(addAbsentClass(onto));
        return mainEntities;
    }
    
    /*Recorre las newEntities creando sus respectivos niveles de jerarquía 
    en el Árbol de Entidades.*/
    private static void addBranchFromEntity(newBranch, OWLClass actualClass,
    newEntities, navigation, onto) {
        for (OWLIntClass enti : newEntities) {
            if (!navigation.contains(enti)) {
                navigation.addLast(enti);
                if (enti.getType().equals("class")) {
                    sub2 = ContentGrouping.generateNewTopics(onto, enti, navigation);
                    addBranchFromEntity(newSubBranch, enti, sub2, navigation, onto);
                }else if (enti.getType().equals("property")) {
                    sub2 = ContentGrouping.generateNewTopics(onto, enti, actualClass);
                    addBranchFromEntity(newSubBranch, actualClass, sub2, navigation, onto);
                }
                navigation.removeLast();
                newBranch.addNewBranchToEntity(enti, newSubBranch);
            }
        }
    }
}
\end{minted}


\section{Implementación Micro Planificación}
\label{apx:impl_micro_plan}

\subsection{Implementación de la verbalización del constructor \emph{OWLObjectComplementOf} en español}
\label{sec:clase_OWLObjectComplementOf}

\begin{minted}[autogobble,linenos]{java}
public class OWLObjectComplementOf extends OWLRestriction{

    public OWLObjectComplementOf(String type, String lang) {
        super(type, lang);
    }

/*La StatementComponent que retorna tiene la lista de Word 'resList',
que representa la oración parcial que se compone en este constructor.
Nótose que a esta lista se le inserta el 'enlace + clase', que se
corresponde con la definición de su gramática.*/

@Override
    protected StatementComponent generateStatementSpanish(TextCotext c) {
        LinkedList<StatementComponent> stmsClasses = new LinkedList<>();
        stmsClasses.add(classes.get(0).generateStatement(c));
        LinkedList<Word> resList = new LinkedList<>();
        LinkedList<Word> enlace = getEnlaceSpanish(stmsClasses.get(0));
        resList.addAll(enlace);
        resList.addAll(stmsClasses.get(0).getListWords());
        StatementComponent stm = new StatementComponentSpanish(c, resList, "ON", type);
        stm.setComplementList(resList);
        return stm;
    }
    
    /*Diferentes enlaces dependiendo el tipo de oracion parcial que recibe*/
    private LinkedList<Word> getEnlaceSpanish(StatementComponent stm) {
        LinkedList<Word> enlace = new LinkedList<>();
        if (stm.getType().equals("T")) {
            WordSpanish lo = new WordSpanish("LO", WordSpanish.TYPE_PRONOUN);
            WordSpanish opuesto = new WordSpanish("OPUESTO", WordSpanish.TYPE_ADJETIVE);
            WordSpanish de = new WordSpanish("DE", WordSpanish.TYPE_PREPOSITION);
            enlace.add(lo);
            enlace.add(opuesto);
            enlace.add(de);
        } else if (stm.getType().equals("SV")) {
            WordSpanish no = new WordSpanish("NO", WordSpanish.TYPE_ADV_NEG);
            enlace.add(no);
        } else if (stm.getType().equals("ON")) {
            WordSpanish ni = new WordSpanish("NI", WordSpanish.TYPE_CONJUNTION_COORD);
            enlace.add(ni);
        }else {
            WordSpanish no = new WordSpanish("EXCEPTO", WordSpanish.TYPE_ADV_NEG);
            enlace.add(no);
        }
        return enlace;
    }
}
\end{minted}


\subsection{Método para generar las Expresiones de Referencia en español}
\label{sec:met_exp_ref}

\begin{minted}[autogobble,linenos]{java}
public LinkedList<Word> getReferenceExpression(TextCotext c) {
    LinkedList<Word> resList = new LinkedList<>();
    Random r = new Random();
    int x = r.nextInt(10);
    /*Tiene un 20% de probabilidad de utilizar como expresión
    de referencia el nombre de su superClase si posee una*/
    if (superClass.size() == 1 && x < 2) {
        TextCotext cot = new TextCotext(null, this, "init", 0, language);
        Word n = superClass.get(0).generateStatement(cot).getNoun();
        String gen = ((WordSpanish) n).getGender();
        resList.add(getPronounDemostrative(gen));
        resList.add(n.getUpercaseWord());
    }else{
        Word noun = statement.getNoun();
        String gen = ((WordSpanish) noun).getGender();
        resList.add(getPronounDemostrative(gen));
    }
    return resList;
}

private WordSpanish getPronounDemostrative(String gen) {
    WordSpanish w;
    if (gen.equals("femenino")) {
        w = new WordSpanish("ESTA", WordSpanish.TYPE_PRONOUN_DEMONSTRATIVE);
    } else {
        w = new WordSpanish("ESTE", WordSpanish.TYPE_PRONOUN_DEMONSTRATIVE);
    }
    return w;
}
\end{minted}

\section{Implementación Realizador}
\label{sec:impl_realizador}
\subsection{Clase \texttt{Section}}
\label{sec:clase_section}

\begin{minted}[autogobble,linenos]{java}
public class Section {
    private LinkedList<OWLIntClass> topics;
    private Title title;
    private LinkedList<Paragraph> paragraphs;
    private LinkedList<Section> subSections;
    private String language;

    public Section(OWLIntClass topic, String lang) {
        init(lang);
        topics.add(topic);
        title = new Title(topic, lang);
        if (topic.getType().equals("class") ||
        topic.getType().equals("individual")) {
            //crear el Paragraph para la informacion de la seccion
            Paragraph inf = new Paragraph(topic, lang);
            paragraphs.add(inf);
        } 
    }
    
    public String generateText() {
    String textF = "";
    textF = titulo de la Section
    textF += contenido de los Parrafos de la Section
    for (Section s: subSections){
        if (s no cumple los criterios para verbalizar como sección){
            textF += contenido de s en forma de párrafo
        }
    }
    /*La realización de las subsecciones se separa en dos
    recorridos distintos para que todos los párrafos estén
    juntos, y luego aparezcan todas las secciones.*/
    for (Section s: subSections){
        if (s cumple los criterios para verbalizar como sección){
            textF += s.generateText();
        }
    }
    
    return textF;
    }
}
\end{minted}



\end{document}
