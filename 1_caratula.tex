%Observación: Si bien la página del prefacio dice que sea empty, debería comenzar alli la numeración. Se sugiere numeración romana. Recomenzar la numeración en el primer capítulo de la tesis  con numeración arábiga.

\titlepage

\begin{center}
\ \\
\ \\
\vspace{-1cm}
 

\ \\

\vspace{0.5cm}
{\Large{\bf \sc Universidad Nacional del Comahue}}\\

\ \\
{\Large { \sc Facultad de Informática}}\\

\vspace{-2.5cm}
\mbox{\hspace{-1cm}\includegraphics[width=2.5cm,height=2.5cm]{img/unc.png}\hspace{13cm} \includegraphics[width=2.5cm,height=2.5cm]{img/fai.png}}


\vspace{6cm}

{\Large {\bf\sc Tesis de Licenciado en Ciencias de la Computación}}\\
\ \\
\ \\
%{\LARGE {\bf Reestructuración de la información \\en el proceso de verbalización}}\\
%{\LARGE {\bf Estrategias a nivel macro y micro en la}}\\
%\vspace{0.3cm}
%{\LARGE {\bf   reestructuración de la información en el }}\\
%\vspace{0.3cm}
%{\LARGE {\bf   proceso de la verbalizacion}}\\


%\vspace{1.3cm}
%{\LARGE {\bf Reestructuración de la información }}\\
%\vspace{0.3cm}
%{\LARGE {\bf a nivel macro y micro   }}\\
%\vspace{0.3cm}
%{\LARGE {\bf   proceso de la verbalizacion}}\\

%\vspace{1.3cm}
{\LARGE {\bf Estrategias para abordar la macro y micro planificación en el proceso de verbalización de ontologías }}\\
%\vspace{0.1cm}
%{\LARGE {\bf   }}\\
%\vspace{0.1cm}
%{\LARGE {\bf     }}\\


\vspace{3cm}
%Reestructuración a nivel macroplanificación y micro en el proceso de verbalización de ontolgías.

%Reestructuración en la tarea de macroplanificación y microplanificación en el proceso de verbalización de ontolgías.

%estrategias de reestructuración de macro y micro estructuras en el proceso de la verbalizacion

%estrategias para abordar la macro y micro planificación en el proceso de verbalización de ontologías

{\Large Martín Bermudez}\\
\vspace{2cm}

{\Large Esp. Sandra Roger}\\
\ \\
%{\Large [Nombre del CoDirector]}\\

\vfill
{\Large {\sc Neuquén}\hspace{6cm}{\sc Argentina}}\\
\ \\

{\Large 2020}\\

\end{center}

\pagebreak

