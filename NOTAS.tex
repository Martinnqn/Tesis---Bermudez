Cosas que hice:
Agregué RDF y Razonamiento de owl en la intro.
Referencié las figuras y subfiguras que aparecen.
%DESPUES LEO ME FALTA LEER EL CAPITULO CONCLUSION Y DEPSUES EMPEIZO A LLEER ESTO.

Cosas que no hice:
En la introducción la palabra Protege[42] se sale del margen y no lo pude arreglar :C
%ya esta

Respecto a la cita de los conceptos usados en los capítulos Organización y Verbalización: todos los conceptos 
(como teoria de veins o centralidad de nodos) los expliqué en el marco teórico, osea en el capítulo de Antecedentes Conceptuales. Cuando los uso en los otros capítulos tengo q hacer referencia a las Secciones donde los explico (PUEDE SER) o al paper origin? No los referencié porque ya los había explicado.
%EN REALIDAD CUALQ DE LAS DOS FORMAS SON VIABLES

Cuando nombro POSTagger en el diseño no digo cual usé, eso lo especifiqué en la implementación. Lo muevo de la implementación al diseño?
%ESPECIFICALO OTRA VEZ .. EN EL DISEÑO VA CON REFERENCIA O PAGINA WEB .. EN LA IMPLEMENTACIÓN PODRIA NO DECIRLO PERO SI ES CITA REPETILO. sI ES PAGINA WEB COO FOOTNOTE NO.


FALTA: hacer una carátula (se la pido a Giuliano).
%AHORA LA PONGO faltaria completar alguans fechas (prefacio ...) y llenar el resto de cosas que ahí te puse. Lo copie de otra tesis asiq hay que revisar de hacer los cambios para tu tesis, creo que lo hice pero revisar.

------------------4/6/20-------------------
Escribí el resumen (que es una copia de la introduccion eliminando algunas cosas y cambiando de lugar otras) pero me parece un poco largo.

%% Resumen: descripci´on de la tesis de hasta 500 palabras. Deber´a contener m´ınimamente motivaci´on, objetivo, resultado y conclusiones. 
%Creo que tiene 460  paalabras aprox  asiq esta bien ..

Estoy en proceso de unificar la nomenclatura de macro y micro planning. tengo que leer todo desde el principio para asegurarme de ser consistente en el uso de los conceptos, porque planificación/planning y planificador son todos validos según el contexto.
%si se que es lo mismo, pero como está como un concepto (hasta resaltado en cursiva) es como que en algun momento hizo ruido leerlo en ingles y en español. Es decir si se pone \textit{Micro-planning} que siempre este igual .. Por ahi segun algun parrafo en particular lo vemos.  

resolví las correcciones de los comentarios.

----------------06/06/20----------------
Unifiqué las palabras micro macro planificacion. Cuando es adecuado hablar de la etapa, utilizo la palabra en español (microplanificacion, macroplanificacion, realizacion). Cuando se habla sobre el módulo que implementa cada etapa, uso microplanificador macroplanificador y realizador.

En 2.3 generacion de lenguaje natural, hay un footnote donde se puede hacer referencia a la tesis de matias.

Leyendo en búsqueda de errores: parte 1.

%termine de leer las conclusiones. un pequeño comentario nada mas. 

-------------07/06/20----------
Tengo q cambiar macro planning por macroplanificacion en la figura 4.3. YA ESTA.

Tengo q modificar la conclusión final evitando la autovaloracion. YA ESTA.

Correji la gramatica 4.6 y otros detalles en otras gramáticas y ejemplos.

hay una frase que no entiendo en 4.4.5 Convención y suposiciones del nombrado de Entidades: "Sin embargo, se considerado como trabajo futuro de esta investigación." Lo dejé comentado por las dudas. 
%OK

Agregué un footnote en la conclusion del capitulo 4 para explicar gramática tradicional y la gramática generativa.
%OK

--------------09/06/20--------------
Modifique las conclusiones del capitulo 4 y la del 5. En el capitulo 4 deje un comentario sobre una afirmacion que no se si está bien usarla. 
%ENEL 4 COMENTE Y PUSE UNA OPCION FIJATE 

----- 11/06/20
estuve viendo y resolviendo algunos comentarios y sancando los que ya estaban resueltos tambien. 

en el cap4 hay una frase que no me gustaba tanto 
\textit{subsection{Enumeración de propiedades, subclases e individuos}
Enumerar esta información es solo para anunciar la existencia de estos componentes, por lo que la sintaxis es simple.
}
la cambie fijate y puse dots. 
hay otro cmabio de un comentario en conclusiones.. sobre una forma verbal fuerte que comentaste

ver el resumen .. algunas oracioens estan raras o poco claras..

bueno ya casi esta arregla eso y completa lo que falta y armamos notas .. ahora las voy haciendo asi las elevamos a Silvia y presentamos la tesis.

%dale ahi correji algunas cosas. Estoy leyendo un trabajo relacionado porque lo habia explicado mal. ya lo correjí.

%En el resumen puse unos comentarios. 
ahi los lei y modifique algo

%Elimine la cursiva de las palabras en español en la conclusion.
